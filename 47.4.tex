\section[Liste der Unterschriften unter der theologischen Erkl�rung][Liste der Unterschriften]{Liste der Unterschriften unter der theologischen Erkl�rung} 
\label{sec:Sirmium351:Teilnehmer}
%\todo{sec:Sirmium351:Teilnehmer}{}
\begin{praefatio}
  \begin{description}
  \item[Anfang 351]Zur Datierung vgl.
    Dok. \ref{sec:Sirmium351:Einberufung}, Einleitung. Die Mehrheit
    der Unterzeichner war schon Teilnehmer der �stlichen Synode von
    Serdica,\esindex{Serdica!a. 343} vgl. \dok{43.13}.
  \item[�berlieferung]Socrates und Sozomenus 
    (\socrhe[i]{II 29,2~f.}; \sozhe[i]{IV 6,4}; vgl. die ausgelassenen
    Textpassagen in Dok. \ref{sec:Sirmium351:Einberufung}) haben die Teilnehmer verschiedener sirmischer Synoden heillos verwirrt. Umgekehrt weist Hilarius\enindex{Hilarius!Bischof von Poitiers}
    seine Liste mit 22 Unterzeichnern f�lschlich der sirmischen Synode
    von 357\esindex{Sirmium!a. 357} zu, deren Beschl�ssen auch Liberius von Rom\enindex{Liberius!Bischof von Rom} (vgl. zu
    Liberius Dok. \ref{sec:ArlesMailand:ep5}) zugestimmt hatte, um
    sein Exil zu beenden, und stellt sie in den Zusammenhang der
    entsprechenden Briefe des Liberius\enindex{Liberius!Bischof von Rom}. Die Liste des Hilarius\enindex{Hilarius!Bischof von Poitiers} kann
    nicht der sirmischen Synode von 357\esindex{Sirmium!a. 357} zugeordnet werden, da f�r
    diese andere Teilnehmer verl��lich bezeugt sind
    (vgl. Dok. \ref{ch:Sirmium357:2Formel}); wegen des 358
    verstorbenen Cecropius\enindex{Cecropius!Bischof von Nikomedien} kann sie aber auch nicht zur sirmischen
    Synode von 359\esindex{Sirmium!a. 359} (Dok. \ref{ch:Sirmium359}) geh�ren. Somit ist sie
    mit gro�er Wahrscheinlichkeit der Synode von 351\esindex{Sirmium!a. 351} zuzuweisen. Zu den Namen
    vgl. \cite[101--103]{Feder1911}.
  \item[Fundstelle]\hilcoll{B VII 9}
    (\cite[170,5--8]{Hil:coll}).
  \end{description}
\end{praefatio}
\begin{pairs}
\selectlanguage{latin}
\begin{Leftside}
%  \beginnumbering
  \specialindex{quellen}{section}{Hilarius!coll.\,antiar.!B VII 9}
  \pstart
\bezeugung{A S H(H\ts{1}H\ts{2}H\ts{3}H\ts{4}) F}\kap{1.}\var{Narcissus}{Narcissus Franciscus \sglat{H\tsub{4}}}\nindex{Narcissus!Bischof von Neronias}
  \pend
  \pstart
  \kap{2.}Theodorus\nindex{Theodorus!Bischof von Heraclea}
  \pend
  \pstart
  \kap{3.}Basilius\nindex{Basilius!Bischof von Ancyra}
  \pend
  \pstart
  \kap{4.}\var{Eudoxius}{Eudosius \sglat{A}}\nindex{Eudoxius!Bischof von Antiochien/Konstantinopel}
  \pend
  \pstart
\kap{5.}Demofilus\nindex{Demophilus!Bischof von Beroea}
  \pend
  \pstart
\kap{6.}\var{Cecropius}{Cecorpius \sglat{F H\tsub{1--3}}
    Cocorpius \sglat{H\tsub{4}}}\nindex{Cecropius!Bischof von Nikomedien}
  \pend
  \pstart
\kap{7.}Silvanus\nindex{Silvanus!Bischof von Tarsus}
  \pend
  \pstart
\kap{8.}\var{Ursacius}{Arsacius \sglat{F H\tsub{1}}}\nindex{Ursacius!Bischof von Singidunum}
  \pend
  \pstart
\kap{9.}Valens\nindex{Valens!Bischof von Mursa}
  \pend
  \pstart
\kap{10.}Euagrius\nindex{Evagrius!Bischof von Mitylene|textit}\nindex{Evagrius!Bischof auf Sizilien|textit}
  \pend
  \pstart
\kap{11.}\var{Hirenius}{Hereneus \sglat{F H\tsub{1}
      H\tsub{2}} Ereneus \sglat{H\tsub{3}} Irenaeus
    \sglat{S\tsub{1}} Henerius \sglat{H\tsub{4}} Hyrenius \sglat{coni. C}}\nindex{Irenaeus!Bischof von Tripolis}
  \pend
  \pstart
\kap{12.}\var{Exuperantius}{Exsuperantius \sglat{F H\tsub{1}\nindex{Exuperantius!Bischof}
      H\tsub{2}}}
  \pend
  \pstart
\kap{13.}Terentianus\nindex{Terentianus!Bischof}
  \pend
  \pstart
\kap{14.}\var{Bassus}{Bustus \sglat{coni. Baronius}}\nindex{Bassus!Bischof von Carpathus}
  \pend
  \pstart
\kap{15.}Gaudentius\nindex{Gaudenius!Bischof von Naissus}
  \pend
  \pstart
\kap{16.}Macedonius\nindex{Macedonius!Bischof von Mopsuestia|textit}\nindex{Macedonius!Bischof von Berytus|textit}
  \pend
  \pstart
\kap{17.}\var{Marcus}{Marthus \sglat{A} Martius
    \sglat{S\tsub{1}}}\nindex{Marcus!Bischof von Arethusa}
  \pend
  \pstart
\kap{18.}\varabb{Acacius\nindex{Acacius!Bischof von Caesarea}}{\sglat{coni. Feder} Actius
    \sglat{A} Atticus \sglat{S\tsub{1} F H}
    Aetius \sglat{susp. Baronius}}
  \pend
  \pstart
\kap{19.}Iulius\nindex{Julius!Bischof}
  \pend
  \pstart
\kap{20.}\var{Surinus}{Serinus \sglat{H\tsub{4}} Severinus
    \sglat{susp. Baronius}}\nindex{Surinus!Bischof}
  \pend
  \pstart
\kap{21.}Simplicius\nindex{Simplicius!Bischof}
  \pend
  \pstart
\kap{22.}\var{et Iunior}{et (ceteri \sglat{S\tsub{1}}) iuniores,
    quibus credidit \sglat{S\tsub{1} F H}}\nindex{Iunior!Bischof}
  \pend
  \endnumbering
\end{Leftside}
\begin{Rightside}
  \begin{translatio}
    \beginnumbering
    \pstart
    Narcissus (von Neronias)\hist{Vgl. \dok[i]{43.13}, Nr. 56.}
    \pend
    \pstart
    Theodorus (von Heraclea)\hist{Vgl. \dok[i]{43.13}, Nr. 10.}
    \pend
    \pstart
    Basilius (von Ancyra)\hist{Vgl. ferner \dok[i]{40.3,1}
    und \dok[i]{43.13}, Nr. 23 und Dok. \ref{ch:Ankyra358}.}
    \pend
    \pstart
    Eudoxius (von Germanicia)\hist{Vgl. \dok[i]{43.13}, Nr. 19.}
    \pend
    \pstart
    Demophilus (von Beroea)\hist{Vgl. \dok[i]{43.13}, Nr. 69.}
    \pend
    \pstart
    Cecropius (von Nikomedien)\hist{Erst Bischof von Laodicaea
      in Phrygien, seit 351 Bischof von Nikomedien
      (\athha[i]{74,5}), 358 bei einem dortigen Erdbeben
      umgekommen (\sozhe[i]{IV 16,5} und Dok. \ref{sec:Rimini359:EpConst},
      Einleitung).}
    \pend
    \pstart
    Silvanus (von Tarsus)\hist{Vgl. Dok. \ref{ch:Kpl360}.1.1,9--12 und Dok. \ref{sec:Kpl360:BriefAbendland}.}
    \pend
    \pstart
    Ursacius (von Singidunum)\hist{Vgl. \dok[i]{43.2,2} Anm. und
      Dok. \ref{ch:Rehabilitierung}.}
    \pend
    \pstart
    Valens (von Mursa)\hist{Vgl. \dok[i]{43.2,2} Anm.; \dok[i]{43.13}, Nr. 73 und Dok. \ref{ch:Rehabilitierung}.}
    \pend
    \pstart
    Evagrius\hist{Vielleicht identisch mit
      Dok. \ref{sec:Seleucia359:Bekenntnis}, Nr. 23 oder mit Evagrius
      von Sizilien, der die Erkl�rung des 
      Basilius von Ancyra\enindex{Basilius!Bischof von Ancyra} u.\,a. an Jovian\enindex{Jovian!Kaiser} 
      363 mit unterschrieb (\socrhe[i]{III 25}).}
    \pend
    \pstart
    Irenaeus (von
    Tripolis)\hist{Vgl. Dok. \ref{sec:Seleucia359:Bekenntnis},
      Nr. 9.} 
    \pend
    \pstart
    Exuperantius
    \pend
    \pstart
    Terentianus
    \pend
    \pstart
    Bassus (von Carpathus)\hist{Vgl. \dok[i]{43.13}, Nr. 55.}
    \pend
    \pstart
    Gaudentius\hist{Vielleicht identisch mit Gaudentius von
      Naissus, vgl. \dok[i]{43.3}, Nr. 4; \doko[i]{43.5}, Nr. 32;
    \doko[i]{43.10}, Nr. 21; \doko[i]{43.11,28} Anm.}
    \pend
    \pstart
    Macedonius (von Mopsuestia?)\hist{Vgl. \dok[i]{43.13}, Nr. 7
      oder Nr. 21 (Bischof von Berytus).}
    \pend
    \pstart
    Marcus (von Arethusa)\hist{Vgl. \dok[i]{43.13}, Nr. 12. Nach
      \socrhe[i]{II 30,1} (irrt�mlich) der Verfasser von Dok. \ref{sec:Sirmium351:1Formel}; 
      er war der Verfasser der theologischen Erkl�rung von Sirmium
      des Jahres 359 (Dok. \ref{ch:Sirmium359:4Formel}).}
    \pend
    \pstart
    Acacius (von Caesarea)\hist{Vgl. \dok[i]{43.13}, Nr. 9,
      Dok. \ref{ch:Seleucia359} und \ref{ch:Kpl360}.}
    \pend
    \pstart
    Julius
    \pend
    \pstart
    Surinus
    \pend
    \pstart
    Simplicius
    \pend
    \pstart
    und Junior
    \pend
    \endnumbering
  \end{translatio}
\end{Rightside}
\Columns
\end{pairs}
\selectlanguage{german}

%%% Local Variables: 
%%% mode: latex
%%% TeX-master: "dokumente"
%%% coding: latin-1
%%% End: 
