\chapter[Synode von Sirmium im Jahr 351 gegen Photin][Synode von Sirmium im Jahr 351]{Synode von Sirmium im Jahr 351 gegen Photin}
\label{ch:Sirmium351}
\begin{praefatio}
  \begin{einleitung}
  \item Da Photin\enindex{Photin!Bischof von Sirmium} trotz seiner Absetzung auf den Synoden
    von Mailand 345\esindex{Mailand!a. 345} (Dok. \ref{sec:Antwortbrief}) und Sirmium 347\esindex{Sirmium!a. 347}
    (Dok. \ref{sec:Sirmium347:Nachrichten}) als Bischof hatte weiter
    amtieren k�nnen, da er wahrscheinlich die Unterst�tzung durch die Bev�lkerung
    Sirmiums\enindex{Sirmium} hatte, und weil Kaiser Constans,\enindex{Constans!Kaiser} obwohl er eigentlich dazu
    verpflichtet war, offenbar kein besonderes Interesse gezeigt
    hatte, die Beschl�sse der Synoden von Mailand\esindex{Mailand!a. 345} und Sirmium\esindex{Sirmium!a. 347}
    durchzusetzen, griff Kaiser Constantius\enindex{Constantius!Kaiser} in die Auseinandersetzung
    ein, nachdem der Usurpator Magnentius\enindex{Magnentius!Kaiser} Kaiser Constans\enindex{Constans!Kaiser} ermordet und
    Constantius\enindex{Constantius!Kaiser} auf dem Weg zur Auseinandersetzung mit Magnentius\enindex{Magnentius!Kaiser} im
    Winter 350/351 sein Hauptquartier in Sirmium\enindex{Sirmium} aufgeschlagen hatte
    (\cite[198]{Seeck:Regesten}). Er berief eine Synode ein, 
    die vornehmlich von Bisch�fen aus dem Osten besucht wurde (vgl.
    die Teilnehmerliste in Dok. \ref{sec:Sirmium351:Teilnehmer}) und auf der
    Photin\enindex{Photin!Bischof von Sirmium} mit Basilius von Ancyra\enindex{Basilius!Bischof von Ancyra} 
    disputierte. Acht kaiserliche Beamte 
    sa�en der Synode als Schiedsrichter vor (vgl. Dok. \ref{sec:Sirmium351:Photin},5) 
    und sechs Stenographen schrieben die Verhandlungen mit 
    (vgl. Dok. \ref{sec:Sirmium351:Photin},6).
    Schlie�lich wurde Photin\enindex{Photin!Bischof von Sirmium} abgesetzt und exiliert; 
    zu seinem Nachfolger wurde Germinius bestimmt\enindex{Germinius!Bischof von Sirmium} (\athha[i]{74,5}).
    Versiegelte Akten wurden Kaiser Constantius\enindex{Constantius!Kaiser} 
    �bermittelt (vgl. Dok. \ref{sec:Sirmium351:Photin},6). \\
    Da die theologische \textit{Ekthesis} dieser Synode (die sog. 1. sirmische
    Formel) in den Anathematismen die Themen der Diskussion auf der
    Synode zwischen Basilius von Ancyra\enindex{Basilius!Bischof von Ancyra} (vgl. zu Basilius \dok[i]{40.3,1}
    und \ref{ch:Ankyra358}) und Photin\enindex{Photin!Bischof von Sirmium} (zur Diskussion
    s. Dok. \ref{sec:Sirmium351:Photin}) aufgreift, scheint diese
    Diskussion entgegen der Darstellung bei \socrhe[i]{II 29~f.}  und
    \sozhe{IV 6} von Photin\enindex{Photin!Bischof von Sirmium} nicht erst im Anschlu� an die Synode aus
    Protest gegen seine Absetzung erbeten worden zu sein, sondern
    zur Synode selbst zu geh�ren. Eventuell wollte Photin\enindex{Photin!Bischof von Sirmium} 
    selbst vor dem Kaiser seine Situation kl�ren (vgl. die Disputation in Dok.
    \ref{sec:Sirmium351:Photin}), der daraufhin eine Synode einberief.  
  \end{einleitung}
  \end{praefatio}
  \section{Berichte �ber die Synode}
\label{sec:Sirmium351:Einberufung}

\begin{praefatio}
\begin{description}
  \item[Anfang 351]Die Datierung der Synode auf Anfang 351 ergibt sich aus der Nennung des
    Postkonsulats des Sergius\enindex{Sergius!Consul} und Nigrinianus\enindex{Nigrinianus!Consul}
    (Dok. \ref{sec:Sirmium351:Einberufung},1,5) und aus der Anwesenheit des
    \textit{Comes} Thalassius\enindex{Thalassius!Comes} als Schiedsrichter
    (Dok. \ref{sec:Sirmium351:Photin},5), der nach seiner
    Ernennung zum \textit{Praefectus Praetorio Orientis} im M�rz 351 nach
    Antiochien\enindex{Antiochien} abreiste (\cite[886 Thalassius
    1]{PLRE}).      
  \item[�berlieferung]�ber die Synode berichten nur \socrhe[i]{II 29} und
    Sozomenus, wobei beide die Berichte �ber die verschiedenen sirmischen Synoden
    jedoch miteinander vermengt und so heillos verwirrt haben. Auf
    Grund dessen ist zu vermuten (vgl.
    \cite[122~f.]{hauschild_antinizaenische_1970}), da� beide hier
    nicht auf die Synagoge des Sabinus\enindex{Sabinus!Bischof von Heraclea} zur�ckgreifen konnten, sondern Sozomenus
    Socrates benutzt hat; da Sozomenus aber �ber die Angaben des
    Socrates hinausgehend aus einer weiteren, nicht mehr zu verifizierenden Quelle einige Nachrichten �ber Photin\enindex{Photin!Bischof von
      Sirmium} bietet, wurde hier nur auf das Zeugnis des Sozomenus
    zur�ckgegriffen. 
  \item[Fundstelle]\sozhe{IV 6,1--4.6}
    (\cite[143,13--25; 144,8--12]{Hansen:Soz}).
  \end{description}
\end{praefatio}

\begin{pairs}
\selectlanguage{polutonikogreek}
\begin{Leftside}
 \beginnumbering
  \specialindex{quellen}{section}{Sozomenus!h.\,e.!IV
    6,1--4.6}
  \pstart
\bezeugung{BC=b}\kap{1}>En to'utw| d`e Fwtein`oc\nindex{Photin!Bischof von Sirmium} 
  t`hn >en Sirm'iw|\nindex{Sirmium} >ekklhs'ian
  >epitrope'uwn, >'hdh pr'oteron kain~hc a<ir'esewc e>ishght`hc
  gen'omenoc, >'eti to~u \nindex{Constantius!Kaiser}\bezeugungpart{basil'ewc}{inc. T} \var{>endhmo~untoc}{>epidhmo~untoc
    \sg{b}} >enj'ade \varomabb{>anafand`on t~w| o>ike'iw| sun'istato
    d'ogmati}{>anafand`on \dots\ d'ogmati}{\sg{T*}}. f'usewc d`e
  >'eqwn e>~u l'egein, ka`i pe'ijein <ikan`oc pollo`uc e>ic t`hn
  <omo'ian \var{<eaut~w|}{a>uto~u \sg{T}} d'oxan >ephg'ageto.
\pend
\pstart
  \kap{2}>'elege d`e <wc je`oc m'en >esti pantokr'atwr e<~ic, <o t~w|
  >id'iw| l'ogw| \var{p'anta}{t`a p'anta \sg{B}} dhmiourg'hsac; t`hn
  d`e pr`o a>i'wnwn g'ennhs'in te ka`i <'uparxin to~u u<io~u o>u
  pros'ieto, >all'' >ek Mar'iac gegen~hsjai t`on Qrist`on e>ishge~ito.
  \kap{3}perip'ustou d`e pollo~ic genom'enou to~u toio'utou d'ogmatoc
  qalep~wc >'eferon o<'i te >ek t~hc d'usewc ka`i t~hc <'ew
  >ep'iskopoi ka`i koin~h| ta~uta newter'izesjai \var{kaj''
    <~wn}{kaj'' <`on \sg{T}} <'ekastoc >ed'oxazen <hgo~unto, kaj'apax
  g`ar t`o diafwno~un >ede'iknuto t~hc Fwteino~u\nindex{Photin!Bischof von Sirmium} p'istewc pr'oc te
  t~wn >en Nika'ia|\sindex{Nicaea!a. 325} t`hn par'adosin jaumaz'ontwn ka`i
  \varom{t~wn}{\sg{b}} t`hn >Are'iou\nindex{Arius!Presbyter in Alexandrien} d'oxan >epaino'untwn.
  \kap{4}>ep`i to'utoic d`e ka`i <o basile`uc\nindex{Constantius!Kaiser} >eqal'epainen; >en d`e
  t~w| t'ote >en \var{Sirm'iw|}{Shrm'iw \sg{T}} diatr'ibwn s'unodon\sindex{Sirmium!a. 351}
  sunek'alese.
  \pend
  \pstart
  \noindent\dots
  \pend
  \pstart
  \kap{5}>epe`i o>~un \varadd{>en}{t~w \sg{T}} Sirm'iw|\nindex{Sirmium} sun~hljon --
  >'etoc d`e to~uto >~hn met`a t`hn Serg'iou\nindex{Sergius!Consul} ka`i
  \varabb{Nigri\Ladd{ni}ano~u}{\sg{coni. Stockhausen} Nigriano~u
    \sg{Soz.(-T)} Nigrhno~u \sg{T}}\nindex{Nigrinianus!Consul} <upate'ian, <hn'ika o>ude`ic
  <'upatoc o>'ute >ek t~hc <'ew o>'ute >ek t~hc d'usewc >anede'iqjh
  di`a t`hn sumb~asan \var{prof'asei}{pr'ofasin \sg{b}} t~wn tur'annwn
  per`i t`a koin`a taraq'hn --, t`on m`en Fwtein`on\nindex{Photin!Bischof von Sirmium} kaje~ilon <wc t`a
  Sabell'iou\nindex{Sabellius} ka`i Pa'ulou to~u Samosat'ewc\nindex{Paulus von Samosata!Bischof von Antiochien} frono~unta.
  \pend
  % \endnumbering
\end{Leftside}
\begin{Rightside}
  \begin{translatio}
    \beginnumbering
    \pstart
    \kapR{1}In dieser Zeit leitete Photin die Kirche in
    Sirmium. Schon fr�her\hist{Vgl. Dok. \ref{ch:Mailand345}.} hatte
    er eine neue H�resie eingef�hrt, und vertrat nun seine Lehre offen, als
    auch der Kaiser hier war.\hist{Bereits in \sozhe{IV 4,2} wird die Anwesenheit des Kaisers
    in Sirmium\enindex{Sirmium} erw�hnt.} Da er die Begabung besa�, gut zu reden,
    und da er f�hig war zu �berzeugen, verf�hrte er viele zu seiner
    Meinung.
    \pend
    \pstart
    \kapR{2}Er sagte, da� es zwar einen allm�chtigen Gott gebe,
    der durch sein eigenes Wort alles geschaffen
    habe,\hist{Vgl. Dok. \ref{sec:Sirmium351:1Formel}, Anathema
      3.14.27.} aber der Zeugung und Existenz des Sohnes vor den
    Zeiten stimmte er nicht zu, sondern f�hrte ein, da� Christus aus
    Maria geboren sei.\hist{Vgl.  Dok. \ref{sec:Sirmium351:1Formel},
      Anathema 5.9.27. Die Zusammenfassung der Lehre Photins bei
      Sozomenus scheint eine vereinfachende Wiedergabe des letzten
      Anathema zu sein.}
    \kapR{3}Nachdem die derartige Lehre bei vielen bekannt geworden war,
    ertrugen dies die Bisch�fe aus dem Westen wie aus dem Osten schwer
    und gemeinsam hielten sie dies f�r eine Neuerung, gerichtet gegen
    den Glauben eines jeden einzelnen, denn der v�llige Widersinn des
    photinischen Glaubens zeigte sich sowohl denen, die die
    niz�nische �berlieferung bewunderten, als auch denen, die die Lehre des
    Arius lobten.
    \kapR{4}Dazu war auch der Kaiser in Unmut geraten. Deshalb
    berief er damals, als er sich in Sirmium aufhielt, eine Synode
    ein.
    \pend
    \pstart
    \noindent\dots
    \pend
    \pstart
    \kapR{6}Nachdem sie\hist{Sozomenus, der hier Socrates folgt, nennt im hier
      ausgelassenen Schlu� von � 4 ebenfalls irrt�mlich Teilnehmer verschiedener sirmischer Synoden.} also in Sirmium
    zusammengekommen waren~-- dies war das Jahr nach dem Konsulat des
    Sergius und Nigrinianus,\hist{\edlabel{fn:Sergius}D.\,h. 351 als Jahr \textit{post c.}  des
      Sergius und Nigrinianus, die 350 Konsuln waren
      (\cite[198]{Seeck:Regesten}).} als weder aus dem Osten noch aus dem
    Westen ein Konsul ernannt worden ist, wegen der das Gemeinwesen
    betreffenden Unruhe, die durch die
    Tyrannen\hist{Mit Tyrannen sind wohl der
      Usurpator Magnentius\enindex{Magnentius!Kaiser} und sein C�sar Decentius\enindex{Magnus Decentius!Caesar}
      (\cite[244~f. Magnus Decentius]{PLRE}) gemeint.} hervorgerufen
    worden sei~--, setzten sie Photin ab, da er die Ansichten des
    Sabellius und des Paulus von
    Samosata\hist{Zu Sabellius vgl. \dok[i]{35.2}; zu Paulus von Samosata vgl. \dok[i]{40}.} teilte.
    \pend
    \endnumbering
  \end{translatio}
\end{Rightside}
\Columns
\end{pairs}
\selectlanguage{german}

%%% Local Variables: 
%%% mode: latex
%%% TeX-master: "dokumente"
%%% coding: latin-1
%%% End: 
