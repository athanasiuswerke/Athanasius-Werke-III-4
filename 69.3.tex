\section{Liste der Unterschriften unter den Tomus ad Antiochenos}
\label{sec:SynAlex362:Unterschriften}
\begin{praefatio}
  \begin{description}
  \item[362]Ein Anhang an den Text des Tomus ad Antiochenos (Dok.
    \ref{sec:SynAlex362:Tomus}) bietet
    zun�chst einen Hinweis auf die anwesenden Bisch�fe, ferner auf die
    abgesandten Diakone des Lucifer,\enindex{Lucifer!Bischof von Calaris} der schon nach Antiochien\enindex{Antiochien}
    abgereist war, und des Paulinus,\enindex{Paulinus!Bischof von Antiochien} sowie einige M�nche des
    Apolinaris,\enindex{Apolinaris!Bischof von Laodicaea} die im Unterschied zu den Diakonen nicht
    stimmberechtigt waren. � 2--3 folgt eine Wiederholung der
    Absender- und Adressatenliste aus dem Pro�mium, erg�nzt um die
    Bischofssitze, wobei Georgius,\enindex{Georg!Bischof} Lucius\enindex{Lucius!Bischof} und Macarius\enindex{Macarius!Bischof} (alle drei
    sonst nicht bekannt) aus unbekannten Gr�nden fehlen.
  \item[�berlieferung]Die Unterschriftenliste ist zusammen mit den Unterschriften und Kommentaren des Eusebius von Vercellae und des
    Paulinus von Antiochien (Dok. \ref{sec:SynAlex362:Euseb} und
    \ref{sec:SynAlex362:Paulinus}) in der \textit{y}-Sammlung der
    Werke des Athanasius als redaktioneller Anhang  an den Tomus ad
    Antiochenos (Dok. \ref{sec:SynAlex362:Tomus}) �berliefert.
  \item[Fundstelle]\athtom{9,3--10,2}
    (\cite[349,8--350,4]{Brennecke2006}).
  \end{description}
\end{praefatio}
\begin{pairs}
\selectlanguage{polutonikogreek}
\begin{Leftside}
  % \beginnumbering
  \specialindex{quellen}{section}{Athanasius!tom.!9,3--10,2}
  \pstart
  \bezeugung{\textit{y} (= BKAOEFM SHG)}
  \kap{1}<Omo'iwc te o<i >'alloi >ep'iskopoi o<i sunelj'ontec
  <up'egrayan ka`i o<i >apostal'entec d`e par`a m`en
  \var{Louk'iferoc}{Loukif'erou \sg{A}}\nindex{Lucifer!Bischof von Calaris} to~u
  >episk'opou Sard'iac n'hsou di'akonoi d'uo,
  \var{<Er'ennioc}{>Er'ennioc \sg{M}}\nindex{Herennius!Diakon} ka`i
  >Agapht'oc,\nindex{Agapetus!Diakon} par`a d`e
  Paul'inou\nindex{Paulinus!Bischof von Antiochien}
  M'aximoc\nindex{Maximus!Diakon} ka`i
  Kal'hmeroc,\nindex{Calemerus!Diakon} ka`i a>uto`i
  di'akonoi. par~hsan d`e \varom{ka'i}{A} tinec
  >Apolinar'iou\nindex{Apolinaris!Bischof von Laodicaea} to~u
  >episk'opou mon'azontec par'' a>uto~u e>ic to~uto pemfj'entec.
  \pend
  \pstart
  \kap{2}\var{>'esti}{>'eti \sg{A}} \varadd{d`e}{ka`i \sg{B}} <'ekastoc t~wn prokeim'enwn >episk'opwn, pr`oc
  o<`uc <h >epistol`h >egr'afh,
  E>us'ebioc\nindex{Eusebius!Bischof von Vercellae} p'olewc
  Birg'illwn t~hc Gall'iac\nindex{Gallien}, Louk'ifer\nindex{Lucifer!Bischof
    von Calaris} t~hc Sard'iac n'hsou,
  >Ast'erioc\nindex{Asterius!Bischof von Petra} Petr~wn t~hc
  >Arab'iac, Kum'atioc\nindex{Cymatius!Bischof von Paltus}
  Palto~u Ko'ilhc Sur'iac,
  >Anat'olioc\nindex{Anatolius!Bischof von Euboea}
  \var{E>ubo'iac}{So'iac A Bero'iac \sg{coni. Tillemont}}.
  \pend
  \pstart
  \kap{3}o<i d`e >episte'ilantec <'o te p'apac
  >Ajan'asioc\nindex{Athanasius!Bischof von Alexandrien} ka`i
  o<i paratuq'ontec s`un a>ut~w| >en >Alexandre'ia|\nindex{Alexandrien} a>ut'oc te
  E>us'ebioc\nindex{Eusebius!Bischof von Vercellae} ka`i
  >Ast'erioc\nindex{Asterius!Bischof von Petra} ka`i o<i
  loipo'i; G'a"ioc\nindex{Gaius!Bischof von Paratonion}
  Paraton'iou t~hc >'eggista Lib'uhc,
  >Agaj`oc\nindex{Agathus!Bischof von Phragonis} Frag'wnewc
  \var{ka`i m'erouc}{<Erma'iwn M} <Elearq'iac t~hc A>ig'uptou,
  >Amm'wnioc\nindex{Ammonius!Bischof von Pachnemunis}
  Paqnemo'unewc ka`i to~u loipo~u m'erouc t~hc <Elearq'iac,
  >Agajoda'imwn\nindex{Agathodaemon!Bischof von Schedia und
    Menela"is} Sqed'iac ka`i Menela'"itou,
  Drak'ontioc\nindex{Dracontius!Bischof von Hermupolis parva}
  <Ermoup'olewc mikr~ac, >Ad'elfioc\nindex{Adelphius!Bischof
    von Onuphis Lychnon} >Ono'ufewc t~hc L'uqnwn,
  <Erm'iwn\nindex{Hermaeon!Bischof von Tanis} T'anewc,
  M'arkoc\nindex{Marcus!Bischof von Zygroi} Z'ugrwn t~hc
  >'eggista Lib'uhc, Je'odwroc\nindex{Theodorus!Bischof von
    Athribis} >Ajr'ibewc, >Andr'eac\nindex{Andreas!Bischof von
    Arsenoites} \var{>Arseno'"itou}{>Arseno'htou \sg{AEH*MG}},
  Pafno'utioc\nindex{Paphnutius!Bischof von Sa"is}
  \var{S'aewc}{S'a"iwc \sg{H}},
  M'arkoc\nindex{Marcus!Bischof von Philoi} F'ilwn,
  Z'w"iloc\nindex{Zo"ilus!Bischof von Andros}
  \var{>'Andrw}{>'Andrwn \sg{M}},
  M~hnac\nindex{Menas!Bischof von Antiphron} >Ant'ifrwn.\kap{}
  \pend
  % \endnumbering
\end{Leftside}
\begin{Rightside}
  \begin{translatio}
    \beginnumbering
    \pstart
    Gleicherma�en unterzeichneten auch die anderen Bisch�fe, die
    zusammengekommen waren, und die Abgesandten von Lucifer, dem
    Bischof von Sardinien, zwei Diakone, Herennius und Agapetus, und
    die Abgesandten von Paulinus, Maximus und Calhemerus, auch sie
    Diakone. Es waren auch einige M�nche des Bischofs Apolinaris
    anwesend, die von ihm dazu geschickt worden waren.
    \pend
    \pstart
    Es ist aber ein jeder der vorliegenden Bisch�fe, an die der Brief
    geschrieben wurde: Eusebius (der Bischof) der Stadt der Virgiller
    in Gallien, Lucifer von Sardinien, Asterius von Petra in Arabien,\hist{Nur in diesem auch sonst nicht fehlerfreien redaktionellen Anhang nach \textit{y} wird Asterius\ als Bischof von Petra in Arabien genannt. Da Petra bis ins 4. Jahrhundert zur Provinz \textit{Arabia} geh�rt hatte und erst im 4. Jahrhundert zur Provinz \textit{Palaestina salutaris}, ist die Zuordnung von Petra zu \textit{Arabia} nicht ungew�hnlich. Unklar ist, ob Asterius mit dem Teilnehmer der Synode von Serdica\esindex{Serdica!a. 343} identisch ist, der dort als arabischer Bischof neben einem Arius von Petra\enindex{Arius!Bischof von Petra} genannt ist (\dok[i]{43}, Einleitung Nr. 14; \doko[i]{43.3}, Nr.54; \doko[i]{43.5}, Nr. 42; \doko[i]{43.9}, Nr. 16).} 
    Cymatius von Paltus in Koile-Syrien, Anatolius von Euboea.
    \pend
    \pstart
    Die Absender aber sind: Der Vater Athanasius und die mit ihm in
    Alexandrien Anwesenden, Eusebius, Asterius und die �brigen: Gaius
    von Paratonion\hist{Gaius von Paratonion in Libyen;
      vgl. \athha[i]{72,4}; \athfuga[i]{7,4}; \athasa{49,3}, Nr. 237.} im
    nahegelegensten Libyen, Agathus von Phragonis und einem Teil der
    Helearchia in �gypten\hist{Agathus von Phragonis und einem Teil
      der Helearchia in �gypten; vgl. \athha[i]{72,4}; \athfuga[i]{7,4};
      \athdraca[i]{7,2}.}, Ammonius von Pachnemunis und dem �brigen Teil
    der Helearchia\hist{Ammonius von Pachnemunis und dem �brigen Teil
      der Helearchia in �gypten; vgl. \athas[i]{49,3}, Nr. 206;
      \athfuga[i]{7,4}; \athhaa[i]{72,4}; \athdraca[i]{7,2}.}, Agathodaimon von
    Schedia und Menelais\hist{Agathodaimon von Schedia und Menelais;
      vgl. \athha[i]{72,4}.}, Dracontius von Hermupolis
    parva\hist{Dracontius von Hermupolis parva;
      vgl. \athdrac (einleitende Anm.); \athfuga[i]{7,4}; \athhaa[i]{72,4}.},
    Adelphius von Onuphis Lychnon\hist{Adelphius von Onuphis Lychnon;
      vgl. \athha[i]{72,4}; \athadelpha; \athfuga[i]{7,4}.}, Hermion von
    Tanis\hist{Hermion von Tanis; ist sonst nicht bekannt.}, Marcus
    von Zygroi im n�chstgelegenen Libyen\hist{Marcus von Zygroi in
      Libyen; vgl. \athha[i]{72,2}; \athfuga[i]{7,4}.}, Theodorus von
    Athribis\hist{Theodorus von Athribis; ist sonst nicht bekannt.},
    Andreas von Arsenoites\hist{Andreas von Arsenoites;
      vgl. \athas[i]{49,39}, Nr. 162.}, Paphnutius von
    Sa"is\hist{Paphnutius von Sa"is; wohl mit einem der drei in
      \athas[i]{49,3} (Nr. 171; Nr. 181; Nr. 226) genannten
      Namenstr�gern identisch; vgl. auch \athhaa[i]{72,4}.},
    Marcus von Philoi\hist{Marcus von Philoi; vgl. \athha[i]{72,2};
      \athfuga[i]{7,4}.}, Zo"ilus von Andros\hist{Zo"ilus von Andros; ist
      sonst nicht bekannt.}, Menas von Antiphron\hist{Menas von
      Antiphron; ist sonst nicht bekannt.}.
    \pend
    \endnumbering
  \end{translatio}
\end{Rightside}
\Columns
\end{pairs}
\selectlanguage{german}


%%% Local Variables:
%%% mode: latex
%%% TeX-master: "dokumente"
%%% coding: latin-1
%%% End:
