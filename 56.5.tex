\section{Fragmente einer anhom�ischen Erkl�rung}
\label{sec:Sirmium358:Arianisch}
%\todoavs{Textkritik und �bersetzung an 58 angleichen}
%\diskussion{Textkritik; �bersetzung �berarbeiten}
\begin{praefatio}
  \begin{description}
  \item[358/359]Diese sechs Fragmente einer anonymen anhom�ischen
    Erkl�rung geh�ren zeitlich zwischen die Synode von Ancyra 358,\esindex{Ancyra!a. 358} auf
    deren Synodalbrief (vgl. auch die Anathematismen 3,5,7,9,11,12,14) 
    sie eventuell reagieren, und den Traktat des Georg von
    Laodicaea\enindex{Georg!Bischof von Laodicaea} und Basilius von 
    Ancyra,\enindex{Basilius!Bischof von Ancyra} in dem sie zitiert werden
    (Dok. \ref{ch:GeorgBasilius359},18--22).
  \item[�berlieferung]Diese Fragmente finden sich in dem nur bei
    Epiphanius �berlieferten Traktat des Georg von Laodicaea\enindex{Georg!Bischof von Laodicaea} und Basilius von 
    Ancyra\enindex{Basilius!Bischof von Ancyra} (Dok. \ref{ch:GeorgBasilius359},18--22). Zur
    �berlieferung vgl. dort. Georg von Laodicaea und Basilius von
    Ancyra nennen keinen Verfasser dieser von ihnen zitierten
    Erkl�rung; \cite[I,184--186]{kopecek_history_1979} und \cite{morales_identification_2007}, haben anhand der inhaltlichen Bez�ge zu Dok. \ref{sec:Aetius-Syntagmation} A"etius\enindex{Aetius!Diakon in Antiochien} als Verfasser dieser Erkl�rung angenommen.
  \item[Fundstelle]\epiph 73,21 (aus
    Dok. \ref{ch:GeorgBasilius359},17--22
    [S. \edpageref{57:59},\lineref{57:59}--\edpageref{57:63},\lineref{57:63}]).
  \end{description}
\end{praefatio}
\begin{pairs}
\selectlanguage{polutonikogreek}
\begin{Leftside}
\specialindex{quellen}{section}{Epiphanius!haer.!73,21}
\beginnumbering
\pstart
\bezeugung{J}\kap{1}t`o g`ar metado~unai <um~in t~wn >en je~w| <rhm'atwn
kall'istwn t`a m'alista di`a braq'ewn proejum'hjhn. <'osoi t`hn kat''
o>us'ian <omoi'othta >apos'wzein t`on u<i`on t~w| patr`i
<upolamb'anousin, >'exw t~hc >alhje'iac beb'hkasi, ((di`a t~hc
((\var{>agen'htou}{>agenn'htou \sg{J}})) proshgor'iac kathgoro~untec to~u
<omo'iou kat'' o>us'ian)).
\pend
\pstart
\begin{small}ka`i p'alin >ero~usin;
\end{small}
\pend
\pstart
\kap{2}<o d`e u<i`oc >el'attwn \varabb{\Ladd{*}}{to~u >agen'htou di`a
  t~hc \sg{susp. Holl}} gen'esewc ka`i >'esti ka`i <wmol'oghtai. o>uk
>'ara t`hn kat'' o>us'ian <omoi'othta \var{>apos'wzei}{>apos'wzein
  \sg{J}} pr`oc t`o \var{>ag'enhton}{>ag'ennhton \sg{J}}, >apos'wzei
d`e >akraifn~h f'erwn >en t~h| o>ike'ia| <upost'asei t`hn to~u jeo~u
bo'ulhsin. o>uko~un <omoi'othta dias'wzei, o>u kat`a t`hn o>us'ian,
>all`a kat`a t`on t~hc jel'hsewc l'ogon, \varabb{\Ladd{*}}{di'oti <o
  je`oc a>ut`on \sg{susp. Holl}, \Ladd{<'oti} \sg{add. Petavius}} o<~ion
>hj'elhsen <upest'hsato.
\pend
\pstart
\begin{small}ka`i p'alin;
\end{small}
\pend
\pstart
\kap{3}p~wc o>uq`i ka`i a>ut'oc moi sunomologe~ic <'oti kat'' o>us'ian o>uk
>'estin <o u<i`oc <'omoioc t~w| patr`i?
\pend
\pstart
\begin{small}ka`i p'alin;
\end{small}
\pend
\pstart
\kap{4}<'otan d`h <o m`en u<i`oc >atele'uthtoc <omologe~itai, o>uk
>ek t~hc o>ike'iac f'usewc t`o z~hn >'eqwn, >all'' >ek t~hc to~u
\var{>agen'htou}{>agenn'htou \sg{J}} >exous'iac, <h d''
\var{>ag'enhtoc}{>ag'ennhtoc \sg{J}} f'usic >ateleut'htwc p'ashc
>exous'iac kre'ittwn >est'i, p~wc \Ladd{\varabb{o>u}{\sg{add. Holl}}}
d~hlo'i e>isin o<i >asebe~ic, <omoi'othti o>us'iac t`o <eteroo'usion
t~hc e>usebe'iac \var{>enall'attontec}{>enal'attontec \sg{J}}
k'hrugma?
\pend
\pstart
\begin{small}ka`i p'alin;
\end{small}
\pend
\pstart
\kap{5}\varabb{di`o}{\sg{Cornarius} di`a \sg{J}} t`o >'onoma
\ladd{\varabb{<o}{\sg{del. Holl}}} ((pat`hr)) o>us'iac o>uk >'esti
dhlwtik'on, >all'' >exous'iac, <uposths'ashc t`on u<i`on pr`o a>i'wnwn
je`on l'ogon, >ateleut'htwc \Ladd{\varabb{>'eqonta}{\sg{add. Holl}}}
\varabb{t`hn}{\sg{susp. Petavius} >~hn \sg{J}}
\ladd{\varabb{>en}{\sg{del. Holl}}} a>ut~w| dwrhje~isan, <`hn >'eqwn
diatele~i o>us'ian te ka`i >exous'ian.
\pend
\pstart
\begin{small}ka`i p'alin;
\end{small}
\pend
\pstart
\kap{6}\Ladd{\varabb{e>i}{\sg{add. Holl}}} o>us'iac e>~inai bo'ulontai t`o
((pat`hr)) dhlwtik'on, >all'' o>uk >exous'iac, t~w|
((\varabb{pat`hr}{\sg{coni. Holl} patr`i \sg{J}})) >on'omati ka`i t`hn to~u
monogeno~uc <up'ostasin prosagoreu'etwsan.
\pend
\endnumbering
\end{Leftside}
\begin{Rightside}
  \begin{translatio}
    \beginnumbering
    \pstart
    Es ist mein gr��ter Wunsch, Euch in Kurzform teilhaben zu lassen
    an den vortrefflichsten Worten in Bezug auf Gott. Alle, die
    annehmen, da� der Sohn die Gleichheit zum Vater in Hinsicht auf das Wesen
    bewahrt, haben sich au�erhalb der Wahrheit gestellt, weil sie
    durch die Bezeichnung �ungeworden� die Gleichheit hinsichtlich des Wesens
    tadeln m��ten.\hist{Vgl. Dok. \ref{sec:Aetius-Syntagmation},1--4.}
    \pend
    \pstart
    \footnotesize{Und wiederum werden sie sagen:}
    \pend
    \pstart
    Der Sohn aber ist und wird als geringer * wegen seiner
    Entstehung bekannt. Er bewahrt
    n�mlich nicht nach dem Wesen die Gleichheit zum
    Ungewordenen, sondern er bewahrt den reinen Willen Gottes, den er rein
    in seiner Existenz tr�gt. Also bewahrt er eine
    Gleichheit, nicht in Hinsicht auf das Wesen, sondern nach der Regel des
    Willens, * denn er (Sohn) existiert so, wie er (Gott) 
    wollte.\hist{Vgl. Eunom., apol. 24.\exindex{Eunomius!apol.!24}}
    \pend
    \pstart
    \footnotesize{Und wiederum werden sie sagen:}
    \pend
    \pstart
    Wieso stimmst nicht auch du mir zu, da� der Sohn dem Wesen nach
    dem Vater nicht gleich ist?
    \pend
    \pstart
    \footnotesize{Und wiederum werden sie sagen:}
    \pend
    \pstart
    Wenn man wohl zustimmt, da� der Sohn endlos ist, wobei er das
    Leben nicht aus der eigenen Natur erh�lt, sondern aus der
    Machtf�lle des Ungewordenen,\hist{Im Hintergrund steht die
      Diskussion um Io 5,26.\ebindex{Johannes!5,26}} wenn aber die ungewordene Natur
    endlos gr��er ist als jede Machtf�lle, wieso sind die dann keine
    offensichtlichen Frevler, die die gottesf�rchtige Verk�ndigung der
    Wesensverschiedenheit durch die Gleichheit des Wesens ersetzen?
    \pend
    \pstart
    \footnotesize{Und wiederum werden sie sagen:}
    \pend
    \pstart
    Deshalb ist der Name �Vater�\hist{Zur Diskussion um den Namen �Vater�
    vgl. Dok. \ref{ch:Sirmium357:2Formel},4f.} nicht ein Hinweis auf ein Wesen,
    sondern auf eine Machtf�lle, die den Sohn vor ewigen Zeiten als Gott Logos
    (\griech{je~ion l'ogon} oder \griech{jeo~u l'ogon}?) ins Dasein
    gerufen hat, der endlos in sich hat, was ihm gegeben wurde,
    das Wesen und die Machtf�lle, die er hat und bis zum Ende beh�lt.
    \pend
    \pstart
    \footnotesize{Und wiederum werden sie sagen:}
    \pend
    \pstart
    Wenn sie wollen, da� der Name �Vater� das Wesen bezeichnet,
    nicht aber die Machtf�lle, dann sollen sie mit dem Namen
    �Vater� auch die Hypostase des Einziggeborenen bezeichnen.
    \pend
    \endnumbering
  \end{translatio}
\end{Rightside}
\Columns
\end{pairs}
\selectlanguage{german}

%%% Local Variables: 
%%% mode: latex
%%% TeX-master: "dokumente"
%%% End: 
