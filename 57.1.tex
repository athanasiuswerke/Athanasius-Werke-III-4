%% Dok. 58
\chapter{Synode von Sirmium im Jahr 359}
\label{ch:Sirmium359}
\begin{praefatio}
\begin{einleitung}
\item Auf Befehl und
    in Gegenwart des Kaisers\enindex{Constantius!Kaiser} traf sich in Sirmium im Mai 359\esindex{Sirmium!a. 359} 
    eine kleine Gruppe offenbar vom Kaiser pers�nlich ausgew�hlter Bisch�fe (die ">Synode"< ist also
    eher eine Kommissionssitzung), 
    um zur Vorbereitung der geplanten Doppelsynode in
    Rimini\esindex{Rimini!a. 359} (vgl. Dok. \ref{ch:Rimini359}) und Seleucia\esindex{Seleucia!a. 359}
    (vgl. Dok. \ref{ch:Seleucia359}) ein Kompromi�papier als
    Grundlage f�r eine theologische Einigung der Kirche des Ostens und des Westens und damit f�r die �berwindung des seit Serdica\esindex{Serdica!a. 343} nicht wirklich �berwundenen Schismas
    zu erarbeiten. Anwesend waren (vgl. den Text unten und die
    Unterschriften in Dok. \ref{sec:Sirmium359:Unterschriften}; 
    vgl. auch \socrhe[i]{II 29} und \sozhe[i]{IV 6,4},
    die der sirmischen Synode von 351\esindex{Sirmium!a. 351} irrt�mlich die Teilnehmer 
    zuweisen, die zu dieser sirmischen Synode 359 geh�ren;
    vgl. ferner \sozhe[i]{IV 16,19}): \\
    Basilius von Ancyra\enindex{Basilius!Bischof von Ancyra},
    Marcus von Arethusa,\enindex{Marcus!Bischof von Arethusa}
    Georg von Alexandrien,\enindex{Georg!Bischof von Alexandrien}
    Pancratius von Pelusium,\enindex{Pancratius!Bischof von Pelusium}
    Hypatianus von Heraclea,\enindex{Hypatianus!Bischof von Heraclea}
    Valens von Mursa,\enindex{Valens!Bischof von Mursa} Ursacius von 
    Singidunum\enindex{Ursacius!Bischof von Singidunum} und Germinius von Sirmium.\enindex{Germinius!Bischof von Sirmium}\\
    Constantius\enindex{Constantius!Kaiser} hatte kirchliche Vertreter zwar des Ostens und des Westens einbestellt, aber theologisch waren ausschlie�lich die von ihm bevorzugten Hom�usianer\enindex{Hom�usianer} und Hom�er,\enindex{Hom�er} deren theologischer Dissens sich hier erst abzuzeichnen beginnt, anwesend.  
    �ber die Verhandlungen sind keine Dokumente �berliefert, aber der R�ckblick
    auf die Ereignisse bei Germinius\enindex{Germinius!Bischof von Sirmium} und die individuellen Unterschriften
    (Dok. \ref{sec:Sirmium359:Unterschriften}) lassen Auseinandersetzungen
    um das richtige Verst�ndnis des Stichwortes \griech{<'omoioc} erkennen.
    Die theologische Erkl�rung unterscheidet sich nur in kleineren Details von
    der Formel, die nach langen Verhandlungen schlie�lich in Konstantinopel 
    im Jahr 360\esindex{Konstantinopel!a. 360} beschlossen wurde (Dok. \ref{sec:Kpl360:Bekenntnis}).
    Die �berlieferung ist bruchst�ckhaft und uneinheitlich.
\end{einleitung}
\end{praefatio}
\section{Bericht des Germinius}
\label{sec:Sirmium359:Germinius}
%\todo{sec:Sirmium359:Germinius}{}
\begin{praefatio}
  \begin{description}
  \item[ca. 366/67]Es handelt sich um die Antwort des Germinius\enindex{Germinius!Bischof von Sirmium} auf den auf den 18. Dezember 366 datierten Brief an Germinius\enindex{Germinius!Bischof von Sirmium} (\textit{ep. Valentis,\enindex{Valens!Bischof von Mursa} Ursacii\enindex{Ursacius!Bischof von Singidunum} et aliorum ad Germinium}). 
  \item[�berlieferung]Dieser knappe R�ckblick auf die Ereignisse von 
  Sirmium im Mai 359\esindex{Sirmium!a. 359} ist in einem sp�teren Brief des Bischofs von Sirmium
    (\textit{ep. ad Rufinianum, Palladium, Severino, Nichae,
      Heliodoro, Romulo, Muciano et Stercorico}; 366/367) enthalten, der wahrscheinlich nicht nur Teilnehmer war, sondern auch traditionell als Ortsbischof den Vorsitz hatte. Germinius\enindex{Germinius!Bischof von Sirmium} verteidigt sich in diesem Brief in der zweiten H�lfte der sechziger Jahre gegen von Valens\enindex{Valens!Bischof von Mursa} und Ursacius\enindex{Ursacius!Bischof von Singidunum} gegen ihn erhobene Vorw�rfe, die theologische  �bereinkunft von Rimini\esindex{Rimini!a. 359} verlassen zu haben.
    Der Brief des Germinius\enindex{Germinius!Bischof von Sirmium} ist nur in
    den coll.\,antiar. des Hilarius von Poitiers\enindex{Hilarius!Bischof von Poitiers} �berliefert und stammt wahrscheinlich aus dem
    zwar zu postulierenden, aber aus den �berlieferten Fragmenten kaum 
    zu rekonstruierenden \textit{Liber III adversus
      Valentem et Ursacium}\exindex{Hilarius!adv. Val. et Urs.!III} des Hilarius von Poitiers, den er nach der R�ckkehr aus dem Osten f�r die Fortf�hrung seiner Polemik gegen Valens\enindex{Valens!Bischof von Mursa} und Ursacius\enindex{Ursacius!Bischof von Singidunum} vermutlich 367/68 zusammengestellt hatte. Unter Umst�nden hat es sich bei diesem zu postulierenden \textit{Liber III adversus
      Valentem et Ursacium} nur um eine unvollendete Zusammenstellung einschl�giger Texte gehandelt.
  \item[Fundstelle]\hilcoll{B VI 3} (\cite[163,12--22]{Hil:coll}).
  \end{description}
\end{praefatio}
\begin{pairs}
\selectlanguage{latin}
\begin{Leftside}
  \beginnumbering
\specialindex{quellen}{section}{Hilarius!coll.\,antiar.!B VI 3}
  \pstart
\bezeugung{A C}
  \kap{}Nam sub bonae memoriae Constantio\nindex{Constantius!Kaiser} imperatore, quando
  inter quosdam coeperat esse de fide dissensio, in conspectu eiusdem
  imperatoris,\nindex{Constantius!Kaiser} praesentibus Georgio\nindex{Georg!Bischof von Alexandrien} 
  episcopo Alexandrinorum ecclesiae,
  \Ladd{\varabb{Pancratio}{\textit{add. Coustant}}
  \varabb{episcopo}{\textit{add. Feder}}} Pelusinorum,\nindex{Pancratius!Bischof von Pelusium}
  Basilio episcopo tunc Anquiritano,\nindex{Basilius!Bischof von Ancyra}
  praesente etiam ipso Valente\nindex{Valens!Bischof von Mursa} et 
  Ursatio\nindex{Ursacius!Bischof von Singidunum} et \varlat{mea}{in ea}{coni. C} 
  \varlat{parvitate}{pravitate}{coni. C},\nindex{Germinius!Bischof von Sirmium} post
  habitam usque in \varabb{noctem}{\textit{\sg{coni. edd.}} 
  nocte \textit{\sg{A}}} de fide disputationem et ad certam regulam
  perductam Marcum\nindex{Marcus!Bischof von Arethusa} ab omnibus nobis \varabb{electum}{\textit{\sg{coni. Faber}} 
  eiectum \textit{\sg{A}}} fidem dictasse, in qua
  fide sic conscriptum est: �filium similem patri per omnia, ut
  sanctae dicunt et docent scripturae,� cuius integrae professioni
  \vartr{omnes consensimus}{consensimus omnes \sglat{coni. C}} 
  et manu nostra suscripsimus.
  \pend
  %\endnumbering
\end{Leftside}
\begin{Rightside}
  \begin{translatio}
    \beginnumbering
    \pstart
    Denn unter Kaiser Constantius, seligen Gedenkens, wurde, als
    unter einigen ein Dissens �ber den Glauben aufkam, unter Aufsicht
    des Kaisers selbst, als Georg, Bischof von Alexandrien,\hist{Georg
      hatte nach Unruhen im Herbst 358
    aus Alexandrien\enindex{Alexandrien} fliehen m�ssen (2. Oktober) und sich offenbar an den
      kaiserlichen Hof in Sirmium\enindex{Sirmium} begeben; er konnte erst am 26. November 361
      nach Alexandrien\enindex{Alexandrien} zur�ckkehren (\athha[i]{2,3.7};
      vgl. \cite[205; 209]{Seeck:Regesten}).} Pancratius von
    Pelusium,\hist{Vgl. Dok. \ref{sec:Seleucia359:Bekenntnis}, Nr.
      27.} Basilius, damals Bischof in Ancyra,\hist{Vgl. \dok[i]{40.3,1};
      Dok. \ref{sec:Sirmium351:Photin}; Dok. \ref{ch:Ankyra358}; Dok. \ref{ch:Seleucia359} und Dok. 
      \ref{sec:Kpl360:Teilnehmer}, Einleitung.} anwesend waren, in
    der Gegenwart auch von Valens und Ursacius selbst und meiner
    Wenigkeit nach einer Diskussion �ber den Glauben bis in die Nacht
    hinein Marcus\hist{Marcus von Arethusa gilt nach Dok. \ref{sec:Sirmium359:Germinius}
      und \sozhe[i]{IV 22,6} als Verfasser der theologischen Erkl�rung.
      Nach \socrhe[i]{II 30,1--4} verfa�te er auch 
      Dok. \ref{sec:Sirmium351:1Formel}, was aber
      unwahrscheinlich ist (vgl. Dok. 
      \ref{sec:Sirmium351:1Formel}, Einleitung), auch wenn er anwesend war 
      (Dok. \ref{sec:Sirmium351:Teilnehmer}, Nr. 17).} von uns allen ausgew�hlt, den Glauben nach einer
    bestimmten Form niederzuschreiben, worin �ber den Glauben
    geschrieben steht: �den Sohn, der dem Vater in allem gleich ist,
    wie es die heiligen Schriften sagen und lehren�. Und wir alle
    stimmten dieser ganzen Erkl�rung zu und unterschrieben mit unserer
    eigenen Hand.
    \pend
    \endnumbering
  \end{translatio}
\end{Rightside}
\Columns
\end{pairs}
\selectlanguage{german}

%%% Local Variables: 
%%% mode: latex
%%% TeX-master: "dokumente"
%%% coding: latin-1
%%% End: 
