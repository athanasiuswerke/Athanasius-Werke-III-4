%%%%%%%%%%%%%%%%%%%%%%%%%%%%%
%%% Konfiguration f�r AW III
%%%%%%%%%%%%%%%%%%%%%%%%%%%%%
%%% Zusammenf�hrung von ledmac.tex und memoir.tex
%%% Mi 25 Apr 2007 23:01:23 CEST  AvS
%%% Umstellung von jurabib auf biblatex
%%% Mi Mai 23 23:09:47 2007 AvS
%%% Umstellung auf biblatex 0.8 und authortitle-avs
%%% Fr Oct 10 2008 14:44:29 Avs
%%%%%%%%%%%%%%%%%%%%%%%%%%%%%
\makeatletter
\usepackage{scrwfile}
%%%%%%%%% Geladene Pakete %%%%%%%%%%%
\usepackage[latin,greek,german]{babel}
\languageattribute{greek}{polutoniko}
\languageattribute{latin}{classic}
\usepackage[latin1]{inputenc}
\usepackage[T1]{fontenc}
\usepackage{etex}
\usepackage{textcomp}
\usepackage{teubner}
\usepackage[parapparatus]{eledmac}
\usepackage{eledpar}
\usepackage{longtable,booktabs}
\usepackage{multicol}
\usepackage{tikz}
\usepackage[autostyle=true,german=guillemets]{csquotes}
\DeclareQuoteAlias[guillemets]{german}{latin}
\DeclareQuoteStyle{polutonikogreek}
    {((}{))}
    {\dt{\flq}}{\dt{\frq}}
%%%%%%% Schriften %%%%%%%%%%%%%%%
\usepackage{AGaramondPro}
%\usepackage[neohellenic]{psgreek}
\usepackage{substitutefont}
\substitutefont{LGR}{\rmdefault}{neohellenic}
\renewcommand{\ttdefault}{pcr}
\renewcommand{\sfdefault}{MyriadProJ}
%%%%%% biblatex-Konfiguration %%%%%%%%%%
\usepackage[backend=biber,language=german,style=authortitle-dw,hyperref,maxnames=4,idembibformat=dash,idemtracker=false,ibidtracker=false]{biblatex}
\bibliography{dokumente}
\defbibheading{edition}{\section*{Editionen}}
\defbibheading{literatur}{\section*{Sekund�rliteratur}}
\renewcommand*{\labelnamepunct}{\addcomma\space}
\renewcommand*{\citenamepunct}{\addcomma\space}
\DeclareFieldFormat{postnote}{#1}
\DeclareFieldFormat{postnote}{#1}
\DeclareFieldFormat{pages}{#1}
\DeclareFieldFormat{pagetotal}{#1}
\DeclareCiteCommand{\editioncite}
  {\usebibmacro{prenote}}
  {\usebibmacro{citeindex}%
   \printnames[labelname]{editor}\addcomma\addspace\usebibmacro{citetitle}}
  {\multicitedelim}
  {\usebibmacro{cite:postnote}}
\DeclareCiteCommand*{\editioncite}%[\mkbibparens]
  {}
  {\printfield{postnote}\addspace}
  {\multicitedelim}
  {\printnames[labelname]{author}}
%%%%%%%%% Hyperref
\usepackage[implicit=true,plainpages=false,pdfpagelabels]{hyperref}
\hypersetup{%
pdftitle = {Dokumente zum arianischen Streit, Lieferung 4},
pdfsubject = {Edition},
pdfkeywords = {Dokumente zum arianischen Streit, Edition, Lieferung 4},
pdfauthor = {\textcopyright\ 2012 Edition Athanasius Werke},
}
\usepackage{memhfixc}
\memhyperindexfalse
\usepackage{microtype}
\title{Dokumente zum arianischen Streit}
\author{Hanns Christof Brennecke\\Uta Heil\\Christian M�ller\\Annette von Stockhausen\\Angelika Wintjes}
\date{\today}
%%%%%%%%%%%% Indices erstellen %%%%%%%%%%%%%%%%%%%%
 % \makeindex[bibel]
 % \makeindex[quellen]
 % \makeindex[stellen]
 % \makeindex[synoden]
 % \makeindex[namen]
%% im Text
\newcommand*{\bindex}[1]{\protect\edindex[bibel]{#1}}
\newcommand*{\sindex}[1]{\protect\edindex[synoden]{#1}}
\newcommand*{\nindex}[1]{\protect\edindex[namen]{#1}}
%% in der Einleitung
\newcommand*{\ebindex}[1]{\protect\index[bibel]{#1}}
\newcommand*{\esindex}[1]{\protect\index[synoden]{#1}}
\newcommand*{\enindex}[1]{\protect\index[namen]{#1}}
\newcommand*{\exindex}[1]{\protect\index[stellen]{#1}}
\let\ledinnotehyperpage\hyperpage
%%%%%%%%% Seitenlayout %%%%%%%%%%%%%
\settrimmedsize{297mm}{210mm}{*}
\setlength{\trimtop}{0pt}
\setlength{\trimedge}{\stockwidth}
\addtolength{\trimedge}{-\paperwidth}
\settypeblocksize{22.5cm}{14.5cm}{*}
\setulmargins{3.5cm}{*}{*}
\setlrmargins{*}{*}{1}
\setmarginnotes{10pt}{11pt}{\onelineskip}
\setheadfoot{\onelineskip}{2\onelineskip}
\setheaderspaces{*}{\onelineskip}{*}
\checkandfixthelayout

%%%%%%%%% Seitenstil (Kopfzeilenlayout) %%%%%%
\makepagestyle{dokumente}
\makepsmarks{dokumente}{%
  \let\@mkboth\markboth
  \def\chaptermark##1{\markboth{\thechapter. \ ##1}{\thechapter. \ ##1}}    % left mark & right marks
  \def\sectionmark##1{\markright{%
    \ifnum \c@secnumdepth>\z@
      \thesection. \ %
    \fi
    ##1}}
  \def\tocmark{\markboth{\contentsname}{\contentsname}}
  \def\lofmark{\markboth{\listfigurename}{\listfigurename}}
  \def\lotmark{\markboth{\listtablename}{\listtablename}}
  \def\bibmark{\markboth{\bibname}{\bibname}}
  \def\indexmark{\markboth{\indexname}{\indexname}}
}
\makeevenhead{dokumente}{\thepage}{\small\leftmark}{}
\makeoddhead{dokumente}{}{\small\rightmark}{\thepage}
% \makeevenfoot{dokumente}{}{\tiny\texttt{\dt{Probeausdruck Athanasius
%       Werke III/4 -- Stand: \today\ -- Blatt \thesheetsequence\ von
%       \thelastsheet\\(\BZR)}}}{}
% \makeoddfoot{dokumente}{}{\tiny\texttt{\dt{Probeausdruck Athanasius
%       Werke III/4 -- Stand: \today\ -- Blatt \thesheetsequence\ von
%       \thelastsheet\\(\BZR)}}}{}

\pagestyle{dokumente}
\aliaspagestyle{chapter}{empty}

%%%%%%%% �berschriften %%%%%%%%%%%%%%
%%%%%%%% Kapitel %%%%%%%%%%%%%%%%%
\makechapterstyle{ath}{
  \renewcommand{\printchaptername}{}
  \renewcommand{\chapternamenum}{}
  \renewcommand{\chaptitlefont}{\centering\Large}
  \renewcommand{\chapnumfont}{\chaptitlefont}
  \renewcommand{\printchapternum}{\centering \chapnumfont \thechapter\space}
  \renewcommand{\afterchaptertitle}{\par\nobreak\vskip \afterchapskip}
}
\chapterstyle{ath}
%%%%%%%% Neuer Kapitelbefehl f�r �Urkunden� %%%
%\let\kapitel\chapter
%\renewcommand{\clearforchapter}{\par}  % Chapter beginnt nicht neue Seite
\newcommand\kapitel{%
  \ifartopt\par\else
%     \clearforchapter
\par
    \thispagestyle{chapter}
    \global\@topnum\z@
  \fi
  \@afterindentfalse
  \@ifstar{\@m@mschapter}{\@m@mchapter}}
%%%%%%%% Sections u.s.w. %%%%%%%%%%%%%%
\setsecheadstyle{\large\centering}
\setsubsecheadstyle{\normalsize\centering}
\setsubsubsecheadstyle{\normalsize\centering}
\setparaheadstyle{\itshape}
%%%%%%%%%%% Abst�nde vor Kapitel etc. %%%
\setlength{\beforechapskip}{2\baselineskip}
\setlength{\midchapskip}{.5\baselineskip}
\setlength{\afterchapskip}{\baselineskip}
\setbeforesecskip{2\baselineskip}
\setbeforeparaskip{0\baselineskip}
\setaftersecskip{\baselineskip}
%%%%%%%% Nummerierung nur bis section %%%%%%
\maxsecnumdepth{section}

%%%%%%% Formatierungen f�r Edition %%%%%%%%
%%%%%%% Handschriftenbeschreibung Siglenliste %%
\newenvironment{codices}%
	{\begin{longtable}[l]{p{1.6cm}p{9cm}p{3cm}}}
	{\end{longtable}\ignorespacesafterend}
\newenvironment{konjektoren}% Angepa�te Tabelleneinstellungen f�r Namen der Konjektoren
	{\begin{longtable}[l]{p{3cm}p{9.5cm}p{0.5cm}}}
	{\end{longtable}\ignorespacesafterend}
\newcommand{\codex}[3]{%
	#1% Sigle
	& #2% Name
	& #3% Datierung
	\\
}
%%%%%%% Paragraphenangabe in nicht-nummeriertem Text (�bersetzungen aus AW III/1+2)
\newcommand{\kapnum}[1]{\noindent\llap{\makebox[.5cm][l]{\footnotesize #1}}\ifthenelse{\equal{#1}{1}}{\noindent}{\quad}}
%%%%%%% Paragraphenangabe im nummerierten Text %%
\newcommand{\kap}[1]{\ledsidenote{\dt{#1}}\bezdummy}
\newcommand{\kapR}[1]{} %% Befehl f�r die Kapitelz�hlung rechts - momentan nicht ausgegeben
%%%%%%%%%%%%%%%% Folio-Angabe von Handschriften am Rand im
%%%%%%%%%%%%%%%% nummerierten Text: Zwei Argumente:
%%%%%%%%%%%%%%%% 1. Handschriftensigle 2. Folio
% \newcommand{\folio}[2]{\dt{ |}\ledsidenote{\dt{\texttt{#1 #2}}}}
%% Ausblenden:
\newcommand{\folio}[2]{}

%%%%%%%%%%%%%%%%%%%%%%%%%%%%%%%%%%%%%%%%%%%%%%%%
%%%%%%% Definition einer Praefatio-Umgebung %%%%
\newenvironment{praefatio}%
{\footnotesize}%
{\ignorespacesafterend}%
%%%%%%%%%%%%% Listen enger gesetzt %%%%%%%%%%%

%%%%%%% Description-Umgebung so umdefiniert, dass folgende Zeilen nicht mehr eingezogen werden
\renewenvironment{description}%
               {\list{}{\labelwidth\z@ \setlength{\leftmargin}{0em}\setlength{\parsep}{0pt}
               \let\makelabel\descriptionlabel}\tightlist}%
               {\endlist\ignorespacesafterend}
% \renewcommand*{\descriptionlabel}[1]{\hspace\labelsep\normalfont\itshape #1}

\newenvironment{einleitung}% allgemeine Einleitung Dok.
               {\list{}{\labelwidth\z@ \setlength{\leftmargin}{0em}\setlength{\parsep}{0pt}}\tightlist}%
               {\endlist\ignorespacesafterend}
%%%%%% �bersetzungs-Umgebung %%%%%%%%%%%
\newenvironment{translatio}
{\small\begin{otherlanguage}{german}}%
{\end{otherlanguage}}

%%%%%% Befehl, um den Autor eines Textst�ckes angeben zu k�nnen %%%
\newcommand{\autor}[1]{%
 \vskip\baselineskip\begin{center}\large{#1}\end{center}%
}

%%%%%%%%%%% Ledmac-Einstellungen
\numberonlyfirstinline
\symlinenum{||}
\preXnotes{1.5em plus .5em minus .5em}
\beforeXnotes{1.5em plus .25em minus .25em}
\Xnotenumfont{\usefont{T1}{AGaramondProJ}{m}{n}}
\renewcommand*{\numlabfont}{\usefont{T1}{AGaramondProJ}{m}{n}\scriptsize}
%%% Bibelstellen
\footparagraph{A}
\inplaceoflemmaseparator[A]{0.25em}
%%% Handschriften
\footparagraph{C}
\nolemmaseparator[C]
\inplaceofnumber[C]{0em}
\inplaceoflemmaseparator[C]{0.25em}
%%% Textkritischer App.
\footparagraph{D}
\inplaceoflemmaseparator[D]{0.25em}
%%% Historischer Kommentar
\renewcommand*{\thefootnoteA}{\alph{footnoteA}}
\footparagraphX{A}
\notenumfontX[A]{\normalfont}
\usepackage{perpage}
\MakePerPage{footnoteA}

%%%%%%%%%% Zeilennummer + 1 = f. (und Zeilennummer l�schen) %%%%%%%%%%%%%%%%%%%
\def\printlines#1|#2|#3|#4|#5|#6|#7|{%
  \begingroup
  \setprintlines{#1}{#2}{#3}{#4}{#5}{#6}%
  \ifnum #1=#4 
  \ifnum #2<#5 
          \newcount\endlinenumber
          \endlinenumber = #5 
          \advance\endlinenumber by -1 
          \ifnum #2<\endlinenumber #2\endashchar#5\else #2\nobreakspace f\fullstop
  \fi
  \fi 
  \ifnum#2=-1 \ledplinenumfalse \fi
  \ifnum #2=#5 \ifledplinenum #2\else \symplinenum\fi
  \fi 
  \fi 
  \ifnum #1<#4 \ifl@d@pnum #1\fullstop\fi #2
  \endashchar
  \ifl@d@pnum #4\fullstop\fi #5
  % \ifl@d@elin \fi
  \fi
  %\ifledplinenum \linenumrep{#2}\else \symplinenum\fi
  % \ifl@d@ssub \fullstop \sublinenumrep{#3}\fi
  % \ifl@d@esl \ifl@d@elin \fullstop\fi \sublinenumrep{#6}\fi
  \endgroup}


%%%%%%% Abk�rzungen f�r den textkritischen Apparat %%
%%%%%%%% Sigle %%%%%%%%%%%%%
%%%%%%%%%%%%%%%%%%%%%%%%%%%%
% Verwendung: \sg{Sigle(ngruppe)} (nur bei lateinischen Buchstaben)
\newcommand{\sg}[1]{\dt{#1}}
% Verwendung: \sglat{Sigle(ngruppe)} (nur bei lateinischen Buchstaben
% und in lateinischem Text)
\newcommand{\sglat}[1]{\dt{\textit{#1}}}
% Verwendung: \sgg{Sigle(ngruppe)} (nur bei griechischen Buchstaben)
\newcommand{\sgg}[1]{\griech{#1}}


%%%%%%%%% Eintrag in Bibelstellenapparat %%%%%%%
%%%%%%%%%%%%%%%%%%%%%%%%%%%%%%%%%%%%%%%%%%%%%%%%
% Verwendung: \bibel{Lemma}{Bibelstelle} Direktes Zitat
\newcommand{\bibel}[2]{\edtext{#1}{\lemma{}\Afootnote[nosep]{\dt{#2}}}}
% Verwendung: \bibelcf{Lemma}{Bibelstelle} Anspielung
\newcommand{\bibelcf}[2]{\edtext{#1}{\Afootnote[nosep]{\dt{vgl. #2}}}}
% bei langen Stellen, auf die verwiesen wird
\newcommand{\bibelcfl}[3]{\edtext{#1}{\lemma{#2}\Afootnote[nosep]{\dt{vgl. #3}}}}
%%%%%%% Textkritische Variante %%%%%%%%%%
%%%%%%%%%%%%%%%%%%%%%%%%%%%%%%%%%%%%%%%%
%%% Lateinische Variante in griechischem Umfeld
\newcommand{\lat}[1]{\dt{\textit{#1}}}
%%% Einfach
% Verwendung: \var{Lemma}{Varianten}
\newcommand{\var}[2]{\edtext{#1}{\Dfootnote{#2}}}

%%% Variante lateinisch - einfach
%% Verwendung \varlat{Lemma}{Variante}{Sigle}
\newcommand{\varlat}[3]{\edtext{#1}{\Dfootnote{\dt{#2} \sglat{#3}}}}

%%% Umstellung
% Verwendung: \vartr{Lemma}{Varianten}
\newcommand{\vartr}[2]{\edtext{#1}{\lemma{}\Dfootnote[nosep]{\responsio\ #2}}}
% Bei lateinischen Texten
\newcommand{\vartrlat}{\vartr}

%%% Auslassung
% Verwendung: \varom{Lemma}{Siglen}
\newcommand{\varom}[2]{\edtext{#1}{\Dfootnote[nosep]{\sg{> #2}}}}
% Verwendung: \varomlat{Lemma}{Siglen}
\newcommand{\varomlat}[2]{\edtext{#1}{\Dfootnote[nosep]{\sg{> \textit{#2}}}}}

%%% Auslassung mit zus�tzlichem Lemma
% Verwendung: \varomabb{Lemma}{Zus�tzliches/abgek�rztes Lemma}{Siglen}
\newcommand{\varomabb}[3]{\edtext{#1}{\lemma{#2}\Dfootnote[nosep]{\sg{> #3}}}}
% lateinischer Text
\newcommand{\varomabblat}[3]{\edtext{#1}{\lemma{#2}\Dfootnote[nosep]{\sg{> \textit{#3}}}}}
%%% Hinzuf�gung
% Verwendung: \varadd{Lemma}{Varianten}
\newcommand{\varadd}[2]{\edtext{#1}{\Dfootnote[nosep]{+ #2}}}

%%% Hinzuf�gung lateinisch
% Verwendung: \varaddlat{Lemma}{hinzugef�gter Texte}{Sigle}
\newcommand{\varaddlat}[3]{\edtext{#1}{\Dfootnote[nosep]{\dt{+ #2} \sglat{#3}}}}

%%% Einfach ohne ]
% Verwendung: \varabb{Lemma}{Varianten}
\newcommand{\varabb}[2]{\edtext{#1}{\Dfootnote[nosep]{#2}}}

%%% Einfach ohne ] lateinisch
\newcommand{\varabblat}[3]{\edtext{#1}{\Dfootnote[nosep]{\dt{#2} \sglat{#3}}}}

%%% Mit zus�tzlichem Lemma
% Verwendung: \varl{Lemma}{Zus�tzliches/abgek�rztes Lemma}{Varianten}
\newcommand{\varl}[3]{\edtext{#1}{\lemma{#2}\Dfootnote{#3}}}

%%% Mit zus�tzlichem Lemma ohne ]
% Verwendung: \varlabb{Lemma}{Zus�tzliches/abgek�rztes Lemma}{Varianten}
\newcommand{\varlabb}[3]{\edtext{#1}{\lemma{#2}\Dfootnote[nosep]{#3}}}

%% Paragraphenz�hlung in Hss
\newcommand{\paracount}[3]{\edtext{#1}{\lemma{}\Dfootnote[nosep]{#2 \sg{#3}}}}

%%%%%%% Bezeugungsleiste %%%%%%%%%%%%%%%%%%%
%% Verwendung: \bezeugung{Handschriften}
\newcommand{\bezdummy}{\edtext{}{\Cfootnote[nonum]{}}} % sorgt f�r
                                % einen leeren Eintrag
\newcommand{\bezeugung}[1]{\txtbeforeXnotes[C]{\dt{#1}\hspace{0.5em}}}
%\newcommand{\bezeugung}[1]{\edtext{}{\Cfootnote[nonum]{\hspace{-0.25em}\sg{#1}}}}
\newcommand{\bezeugungpart}[2]{\edtext{#1}{\Cfootnote{\sg{#2}}}}
\newcommand{\bezeugungpartlang}[3]{\edtext{#1}{\lemma{#2}\Cfootnote[nosep]{\sg{#3}}}}
\newcommand{\bezeugungextra}[1]{\edtext{}{\Cfootnote[nonum]{\hspace{-0.8em}\sg{#1}}}} 
%%%%%%%% Historischer Kommentar %%%%%%%%%%%%
%%%%%%%%%%%%%%%%%%%%%%%%%%%%%%%%%%%%%%%%%%%%%
%% Verwendung: \hist{Kommentar}
\newcommand{\hist}[1]{\footnoteA{#1}}

%%%%%%%%%%%%%% Weitere Abk�rzungen %%%%%%%%%%%
%%%%%%%%%%%%%%%%%%%%%%%%%%%%%%%%%%%%%%%%%%%%%
\newcommand{\dt}{\foreignlanguage{german}}
\newcommand{\griech}{\foreignlanguage{polutonikogreek}}
\newcommand{\latein}[1]{\foreignlanguage{latin}{\textit{#1}}}
\newcommand{\mg}{\textsuperscript{\,mg}}
\newcommand{\slin}{\textsuperscript{\,sl}}
\newcommand{\corr}{\textsuperscript{\,c}}
\newcommand{\ras}{\textsuperscript{\,ras}}
\newcommand{\ts}{\textsuperscript}
\newcommand{\tsub}{\textsubscript}
%%%%%% Neudefinition von teubner.sty-Abk�rzungen %%%%%%%%%%
\renewcommand{\Ladd}[1]{\dt{<}#1\dt{>}}%
\renewcommand{\ladd}[1]{{[}%
    {#1\/}{]}}%
\renewcommand{\dBar}{\textbardbl}
%%%%%% Linie �ber Buchstaben %%%%%%%%%%
\usepackage{ulem}
\protected\def\oline{\bgroup \ULdepth=-1.9ex \ULset}


%%%%%%%%% Referenzierung von Textpassagen %%%%%%%%%%%%%%%
%% Dirk-Jan Dekker in comp.text.tex: cross-referencing entire passages in ledmac
\newcommand{\refpassage}[2]{%
   \xpageref{#1},\xlineref{#1}%
   \ifnum\xpageref{#1}=\xpageref{#2}
      \ifnum\xlineref{#1}=\xlineref{#2}
      \else
         \endashchar\xlineref{#2}%
      \fi
   \else
      \endashchar\xpageref{#2},\xlineref{#2}%
   \fi
}

%%%%%%%%%%%%%%%%%%%%%%%%%%%%%%%%%%%%%%%%%%%%%%%%%
%%%%%%%% Ma�e und Einstellungen %%%%%%%%%%%%%%%%%%%
\lineation{page}
\sidenotemargin{left}
\linenummargin{right}
\setlength{\linenumsep}{.5em}
\setlength{\parindent}{1em}
\setlength\parskip{0em} 
   \tolerance1414
   \hbadness1414
   \vbadness1414
\emergencystretch35pt
   \hfuzz0.5pt
\widowpenalty=10000
   \vfuzz \hfuzz
 \clubpenalty=10000

%%%%%%%%%%%%%%%%%%%%%%%%%%%%%%%%%
%%%%%%% Paralleltext  %%%%%%%%%%%%
\maxchunks{300}
\renewcommand*{\Rlineflag}{}
% \renewcommand*{\leftlinenum}{}
% \renewcommand*{\rightlinenum}{}
\renewcommand*{\leftlinenumR}{}
\renewcommand*{\rightlinenumR}{}
%%%%%%% Spaltenabstand %%%%%%%%%%
\setlength{\Lcolwidth}{.46\textwidth}
\setlength{\Rcolwidth}{.50\textwidth}
\renewcommand*{\columnseparator}{\hspace{.04\textwidth}}
%\setlength{\Lcolwidth}{65mm}
%\setlength{\Rcolwidth}{75mm}

%%%%%%%%%%%%%%%%%%%%%%%%%%%%%%%
%%%%%%% TOC %%%%%%%%%%%%%%%%%%%%%
%%%%%%% Style des Titels u. d. Kapitel�berschriften %%%
\renewcommand{\printtoctitle}[1]{\centering\normalfont\Large #1}
\renewcommand{\cftchapterfont}{\raggedright\normalfont\large}
\renewcommand{\cftchapterpagefont}{\normalfont}
%%%%%%% Abst�nde zwischen Nummer und Titel
\setlength{\cftchapternumwidth}{3em}
\setlength{\cftsectionnumwidth}{3em}
\setlength{\cftsectionindent}{0em}
%%%%%%% Punkte %%%%%%%%%%%%%%%%%%%%
\renewcommand{\cftsectiondotsep}{\cftdotsep}
\renewcommand{\cftchapterdotsep}{\cftdotsep}


%%%%%%%%%%%%%%%%%%%%%%%%%%%%%%%%%%%%%%
%%%%%%% Captions %%%%%%%%%%%%%%%%%%%
\captionnamefont{\scriptsize}
\captiontitlefont{\scriptsize}

%%%%%%%%%%%%%%%%%%%%%%%%%%%%%%%%%%%
%%%%%%% Index %%%%%%%%%%%%%%%%%%%%%
\renewcommand{\pagelinesep}{,} % Trenner zw. Seite u. Zeile
\renewcommand*{\see}[2]{\textit{\seename} #1}
%%%%%%% Dreispaltig und sonstige Index-Formatierungen 
\setlength{\indexcolsep}{1cm}
% \setlength{\indexrule}{.1pt}
\renewenvironment{theindex}{%
  \begin{multicols}{3}[\section*{\indexname}\preindexhook][10\baselineskip]%
 \raggedcolumns
  \raggedright\footnotesize
  \indexmark
  \setlength{\columnseprule}{\indexrule}
  \setlength{\columnsep}{\indexcolsep}
  \ifnoindexintoc\else
    \phantomsection
    \addcontentsline{toc}{section}{\indexname}
  \fi
%   \thispagestyle{chapter}
  \parindent\z@
  \parskip\z@ \@plus .3\p@\relax
  \let\item\@idxitem}
{\end{multicols}}
% \renewcommand{\@idxitem}  {\par\hangindent 20\p@}
% \renewcommand{\subitem}   {\par\hangindent 40\p@}
% \renewcommand{\subsubitem}{\par\hangindent 60\p@}
% \renewcommand{\indexspace}{\par \vskip 20\p@ \@plus5\p@ \@minus3\p@\relax}


%%%%%%%%%%%%%%%%%%%%%%%%%%%%%%%%%%%%%%%%%
%%%%%%% Diskussionsbedarf / Noch zu erledigen %%%%%%%%%%%%%%%
% \usepackage[german,colorinlistoftodos]{todonotes}
\usepackage[disable]{todonotes} % Option: disable
\newcommand{\todohcb}[1]{\todo[author=HCB,inline,color=green!40]{#1}}
\newcommand{\todohiwi}[1]{\todo[author=Hiwi,inline,color=red!40]{#1}}
\newcommand{\todoavs}[1]{\todo[author=AvS,inline,color=blue!20]{#1}}
\newcommand{\todocm}[1]{\todo[author=CM,inline,color=yellow!40]{#1}}
\newcommand{\todoun}[1]{\todo[author=??,inline,color=orange!40]{#1}}
\newcommand{\todotk}[1]{\todo[author=TK�,inline,color=orange!60]{#1}}
\newcommand{\diskussion}[1]{\todo[inline]{#1}}
\newcommand{\diskussionsbedarf}{\begin{framed}\bfseries\centering Diskussionsbedarf\end{framed}}
\newcommand{\frage}[1]{\protect\addcontentsline{tdo}{todo}{#1}\LitNil\LitNil\texttt{#1}\LitNil\LitNil\ }
% \newcommand{\diskussionsbedarf}[1]{} % Umdefinition, falls Diskussionsbedarf nicht angezeigt werden soll
% \newcommand{\frage}[1]{}

%%%%%%%%%%%%%% Stellenangaben %%%%%%%%%%%%%
\usepackage{xifthen}
% % Autor, Buch Stelle oder (leer) Buch Stelle
% \newcommand{\stelle}[4][]{% allgemeiner Befehl
%   \ifthenelse{\isempty{#1}}{\ifthenelse{\isempty{#2}}{#3 #4}{#2, #3
%   #4}}{\ifthenelse{\isempty{#2}}{#3 #4\exindex{#3!#4}}{#2, #3
%   #4\exindex{#2!#3!#4}}}}
% -o-Befehl: ohne Autor, Buch (bei mehreren Verweisen hintereinander)
% -a-Befehl: ohne Autor (bei Verweis ohne Nennung des Autors, nur Werk)
%% Dokumente
\newcommand{\dok}[2][]{%
  \ifthenelse{\isempty{#1}}{Dok. #2}{Dok. #2\exindex{Dokumente!#2}}}
\newcommand{\doko}[2][]{%
  \ifthenelse{\isempty{#1}}{#2}{#2\exindex{Dokumente!#2}}}
%% Markell
\newcommand{\markell}[2][]{%
  \ifthenelse{\isempty{#1}}{Markell, fr. #2}{Markell, fr. #2\exindex{Markell!fr.!#2}}}
\newcommand{\markello}[2][]{%
  \ifthenelse{\isempty{#1}}{#2}{#2\exindex{Markell!fr.!#2}}}
\newcommand{\markella}[2][]{%
  \ifthenelse{\isempty{#1}}{fr. #2}{fr. #2\exindex{Markell!fr.!#2}}}
%% Euseb
\newcommand{\eushe}[2][]{%
  \ifthenelse{\isempty{#1}}{Eus., h.\,e. #2}{Eus., h.\,e. #2\exindex{Eusebius!h.e.!#2}}}
\newcommand{\eusvc}[2][]{%
  \ifthenelse{\isempty{#1}}{Eus., v.\,C. #2}{Eus., v.\,C. #2\exindex{Eusebius!v.C.!#2}}}
\newcommand{\eusde}[2][]{%
  \ifthenelse{\isempty{#1}}{Eus., d.\,e. #2}{Eus., d.\,e. #2\exindex{Eusebius!d.e.!#2}}}
%% Sokrates
\newcommand{\socrhe}[2][]{%
  \ifthenelse{\isempty{#1}}{Socr., h.\,e. #2}{Socr., h.\,e. #2\exindex{Socrates!h.e.!#2}}}
\newcommand{\socrheo}[2][]{%
  \ifthenelse{\isempty{#1}}{#2}{#2\exindex{Socrates!h.e.!#2}}}
%% Sozomenos
\newcommand{\sozhe}[2][]{%
  \ifthenelse{\isempty{#1}}{Soz., h.\,e. #2}{Soz., h.\,e. #2\exindex{Sozomenus!h.e.!#2}}}
\newcommand{\sozheo}[2][]{%
  \ifthenelse{\isempty{#1}}{#2}{#2\exindex{Sozomenus!h.e.!#2}}}
%% Theodoret
\newcommand{\thhe}[2][]{%
  \ifthenelse{\isempty{#1}}{Thdt., h.\,e. #2}{Thdt., h.\,e. #2\exindex{Theodoret!h.e.!#2}}}
\newcommand{\thheo}[2][]{%
  \ifthenelse{\isempty{#1}}{#2}{#2\exindex{Theodoret!h.e.!#2}}}
%% Euagrios
\newcommand{\evagr}[2][]{%
  \ifthenelse{\isempty{#1}}{Evagr., h.\,e. #2}{Evagr., h.\,e. #2\exindex{Evagrius!h.e.!#2}}}
%% Rufin
\newcommand{\rufhe}[2][]{%
  \ifthenelse{\isempty{#1}}{Rufin., hist. #2}{Rufin., hist. #2\exindex{Rufinus!hist.!#2}}} %TLL
\newcommand{\rufheo}[2][]{%
  \ifthenelse{\isempty{#1}}{#2}{#2\exindex{Rufinus!hist.!#2}}} %TLL
%% Philostorgios
\newcommand{\philost}[2][]{%
  \ifthenelse{\isempty{#1}}{Philost., h.\,e. #2}{Philost., h.\,e. #2\exindex{Philostorgius!h.e.!#2}}}
\newcommand{\philosto}[2][]{%
  \ifthenelse{\isempty{#1}}{#2}{#2\exindex{Philostorgius!h.e.!#2}}}
%% Hilarius
\newcommand{\hilcoll}[2][]{%
  \ifthenelse{\isempty{#1}}{Hil., coll.\,antiar. #2}{Hil., coll.\,antiar. #2\exindex{Hilarius!coll. antiar.!#2}}} %TLL
\newcommand{\hilcollo}[2][]{%
  \ifthenelse{\isempty{#1}}{#2}{#2\exindex{Hilarius!coll. antiar.!#2}}} %TLL
\newcommand{\hilcolla}[2][]{%
  \ifthenelse{\isempty{#1}}{coll.\,antiar. #2}{coll.\,antiar. #2\exindex{Hilarius!coll. antiar.!#2}}} %TLL
\newcommand{\hilsyn}[2][]{%
  \ifthenelse{\isempty{#1}}{Hil., syn. #2}{Hil., syn. #2\exindex{Hilarius!syn.!#2}}} %TLL
\newcommand{\hilsyno}[2][]{%
  \ifthenelse{\isempty{#1}}{#2}{#2\exindex{Hilarius!syn.!#2}}} %TLL
\newcommand{\hilsyna}[2][]{%
  \ifthenelse{\isempty{#1}}{syn. #2}{syn. #2\exindex{Hilarius!syn.!#2}}} %TLL
\newcommand{\hiladconst}[2][]{%
  \ifthenelse{\isempty{#1}}{Hil., ad Const. #2}{Hil., ad Const. #2\exindex{Hilarius!ad Const.!#2}}}
\newcommand{\hiladconsta}[2][]{%
  \ifthenelse{\isempty{#1}}{ad Const. #2}{ad Const. #2\exindex{Hilarius!ad Const.!#2}}}
\newcommand{\hilconst}[2][]{%
  \ifthenelse{\isempty{#1}}{Hil., c.\,Const. #2}{Hil., c.\,Const. #2\exindex{Hilarius!c. Const.!#2}}}
\newcommand{\hilconsta}[2][]{%
  \ifthenelse{\isempty{#1}}{c.\,Const. #2}{c.\,Const. #2\exindex{Hilarius!c. Const.!#2}}}
\newcommand{\hilconsto}[2][]{%
  \ifthenelse{\isempty{#1}}{#2}{#2\exindex{Hilarius!c. Const.!#2}}}
\newcommand{\hilaux}[2][]{%
  \ifthenelse{\isempty{#1}}{Hil., c.\,Aux. #2}{Hil., c.\,Aux. #2\exindex{Hilarius!c. Aux.!#2}}} %TLL
%% Athanasius
\newcommand{\athar}[2][]{%
  \ifthenelse{\isempty{#1}}{Ath., Ar. #2}{Ath., Ar. #2\exindex{Athanasius!Ar.!#2}}}
\newcommand{\athdrac}[2][]{%
  \ifthenelse{\isempty{#1}}{Ath., ep.\,Drac. #2}{Ath., ep.\,Drac. #2\exindex{Athanasius!ep. Drac.!#2}}}
\newcommand{\athdraca}[2][]{%
  \ifthenelse{\isempty{#1}}{ep.\,Drac. #2}{ep.\,Drac. #2\exindex{Athanasius!ep. Drac.!#2}}}
\newcommand{\athadelpha}[2][]{%
  \ifthenelse{\isempty{#1}}{ep.\,Adelph. #2}{ep.\,Adelph. #2\exindex{Athanasius!ep. Adelph.!#2}}}
\newcommand{\athmort}[2][]{%
  \ifthenelse{\isempty{#1}}{Ath., ep.\,mort.\,Ar. #2}{Ath., ep.\,mort.\,Ar. #2\exindex{Athanasius!ep. mort. Ar.!#2}}}
\newcommand{\athmorta}[2][]{%
  \ifthenelse{\isempty{#1}}{ep.\,mort.\,Ar. #2}{ep.\,mort.\,Ar. #2\exindex{Athanasius!ep. mort. Ar.!#2}}}
\newcommand{\athepict}[2][]{%
  \ifthenelse{\isempty{#1}}{Ath., ep.\,Epict. #2}{Ath., ep.\,Epict. #2\exindex{Athanasius!ep. Epict.!#2}}}
\newcommand{\athjov}[2][]{%
  \ifthenelse{\isempty{#1}}{Ath., ep.\,Jov. #2}{Ath., ep.\,Jov. #2\exindex{Athanasius!ep. Jov.!#2}}}
\newcommand{\athjovo}[2][]{%
  \ifthenelse{\isempty{#1}}{#2}{#2\exindex{Athanasius!ep. Jov.!#2}}}
\newcommand{\athafr}[2][]{%
  \ifthenelse{\isempty{#1}}{Ath., ep.\,Afr. #2}{Ath., ep.\,Afr. #2\exindex{Athanasius!ep. Afr.!#2}}}
\newcommand{\athaeg}[2][]{%
  \ifthenelse{\isempty{#1}}{Ath., ep.\,Aeg.\,Lib. #2}{Ath., ep.\,Aeg.\,Lib. #2\exindex{Athanasius!ep. Aeg. Lib.!#2}}}
\newcommand{\athha}[2][]{%
  \ifthenelse{\isempty{#1}}{Ath., h.\,Ar. #2}{Ath., h.\,Ar. #2\exindex{Athanasius!h. Ar.!#2}}}
\newcommand{\athhaa}[2][]{%
  \ifthenelse{\isempty{#1}}{h.\,Ar. #2}{h.\,Ar. #2\exindex{Athanasius!h. Ar.!#2}}}
\newcommand{\athhao}[2][]{%
  \ifthenelse{\isempty{#1}}{#2}{#2\exindex{Athanasius!h. Ar.!#2}}}
\newcommand{\athfug}[2][]{%
  \ifthenelse{\isempty{#1}}{Ath., fug. #2}{Ath., fug. #2\exindex{Athanasius!fug.!#2}}}
\newcommand{\athfuga}[2][]{%
  \ifthenelse{\isempty{#1}}{fug. #2}{fug. #2\exindex{Athanasius!fug.!#2}}}
\newcommand{\athas}[2][]{%
  \ifthenelse{\isempty{#1}}{Ath., apol.\,sec. #2}{Ath., apol.\,sec. #2\exindex{Athanasius!apol. sec.!#2}}}
\newcommand{\athasa}[2][]{%
  \ifthenelse{\isempty{#1}}{apol.\,sec. #2}{apol.\,sec. #2\exindex{Athanasius!apol. sec.!#2}}}
\newcommand{\athsyn}[2][]{%
  \ifthenelse{\isempty{#1}}{Ath., syn. #2}{Ath., syn. #2\exindex{Athanasius!syn.!#2}}}
\newcommand{\athsyna}[2][]{%
  \ifthenelse{\isempty{#1}}{syn. #2}{syn. #2\exindex{Athanasius!syn.!#2}}}
\newcommand{\athsyno}[2][]{%
  \ifthenelse{\isempty{#1}}{#2}{#2\exindex{Athanasius!syn.!#2}}}
\newcommand{\athtom}[2][]{%
  \ifthenelse{\isempty{#1}}{Ath., tom. #2}{Ath., tom. #2\exindex{Athanasius!tom.!#2}}}
\newcommand{\athac}[2][]{%
  \ifthenelse{\isempty{#1}}{Ath., apol.\,Const. #2}{Ath., apol.\,Const. #2\exindex{Athanasius!apol. Const.!#2}}}
\newcommand{\athdion}[2][]{%
  \ifthenelse{\isempty{#1}}{Ath., Dion. #2}{Ath., Dion. #2\exindex{Athanasius!Dion.!#2}}}
\newcommand{\athdiono}[2][]{%
  \ifthenelse{\isempty{#1}}{#2}{#2\exindex{Athanasius!Dion.!#2}}}
\newcommand{\athinc}[2][]{%
  \ifthenelse{\isempty{#1}}{Ath., inc. #2}{Ath., inc. #2\exindex{Athanasius!inc.!#2}}}
\newcommand{\athinca}[2][]{%
  \ifthenelse{\isempty{#1}}{inc. #2}{inc. #2\exindex{Athanasius!inc.!#2}}}
\newcommand{\athinco}[2][]{%
  \ifthenelse{\isempty{#1}}{#2}{#2\exindex{Athanasius!inc.!#2}}}
\newcommand{\athdecr}[2][]{%
  \ifthenelse{\isempty{#1}}{Ath., decr. #2}{Ath., decr. #2\exindex{Athanasius!decr.!#2}}}
\newcommand{\athdecro}[2][]{%
  \ifthenelse{\isempty{#1}}{#2}{#2\exindex{Athanasius!decr.!#2}}}
%% Euseb
\newcommand{\euseth}[2][]{%
  \ifthenelse{\isempty{#1}}{Eus., e.\,th. #2}{Eus., e.\,th. #2\exindex{Eusebius!e. th.!#2}}}
\newcommand{\eusetho}[2][]{%
  \ifthenelse{\isempty{#1}}{#2}{#2\exindex{Eusebius!e. th.!#2}}}
%% Epiphanius
\newcommand{\epiph}[2][]{%
  \ifthenelse{\isempty{#1}}{Epiph., haer. #2}{Epiph., haer. #2\exindex{Epiphanius!haer.!#2}}}
\newcommand{\sulpchr}[2][]{%
  \ifthenelse{\isempty{#1}}{Sulp.\,Sev., chron. #2}{Sulp.\,Sev.,
    chron. #2\exindex{Sulpicius Severus!chron.!#2}}}
\newcommand{\sulpchra}[2][]{%
  \ifthenelse{\isempty{#1}}{chron. #2}{chron. #2\exindex{Sulpicius Severus!chron.!#2}}}
\newcommand{\marvict}{Mar. Victorin., adv. Arium } %TLL
\newcommand{\mortpers}{Lact., mort.\,pers. } %TLL
\newcommand{\tertjud}{Tert., adv.\,Iud. }  %TLL
\newcommand{\tertapol}{Tert., apol. } %TLL
\newcommand{\tertcor}{Tert., coron. }  %TLL
\newcommand{\tertidol}{Tert., idol. }  %TLL
\newcommand{\lactinst}{Lact., inst. } %TLL
\newcommand{\lactira}{Lact., ira } %TLL
\newcommand{\virill}[2][]{%
  \ifthenelse{\isempty{#1}}{Hier., vir.\,ill. #2}{Hier., vir.\,ill. #2\exindex{Hieronymus!vir. ill.!#2}}} %TLL
\newcommand{\hiercl}[2][]{%
  \ifthenelse{\isempty{#1}}{Hier., c.\,Lucif. #2}{Hier., c.\,Lucif. #2\exindex{Hieronymus!c. Lucif.!#2}}} %TLL
\newcommand{\hierchron}{Hier., chron. }  %TLL
\newcommand{\tacann}{Tac., ann. } %TLL
\newcommand{\did}{Did. }
\newcommand{\ambrfid}{Ambr., fid. } %TLL
\newcommand{\augep}{Aug., epist. } %TLL
\newcommand{\orcels}{Or., Cels. }
\newcommand{\zach}{Zach. Mit., h.\,e. }%Zacharias Rhetor
\newcommand{\anagn}{Thdr. Lect., h.\,e. }
\newcommand{\thphn}{Thphn., chron. }
\newcommand{\libprec}[2][]{%
  \ifthenelse{\isempty{#1}}{Faustin., lib.\,prec. #2}{Faustin.,
    lib.\,prec. #2\exindex{Faustinus!lib. prec.!#2}}}
\newcommand{\libpreco}[2][]{%
  \ifthenelse{\isempty{#1}}{#2}{#2\exindex{Faustinus!lib. prec.!#2}}}
\newcommand{\basep}[2][]{%
  \ifthenelse{\isempty{#1}}{Bas., ep. #2}{Bas.,
    ep. #2\exindex{Basilius!ep.!#2}}}
\newcommand{\basepo}[2][]{%
  \ifthenelse{\isempty{#1}}{#2}{#2\exindex{Basilius!ep.!#2}}}
%% Search & replace regexp
%% \\hilcoll\s-\([AB]\s-[IV]+\s-[0-9][,.]?[0-9]?[.,]?[0-9]*[-~f.]*[0-9]*\) -> \\hilcoll{\1}
%% \\athha\s-\([0-9]+\s.[0-9]*\) -> \\athha{\1}
%% \\socrhe\s-\([IV]+\)\s-\([0-9]+[.,]?[0-9]*[-~f.]*[0-9]*\) -> \\socrhe{\1 \2}
%% Dok.\s-\([0-9]+[.]?[0-9]?[,]?[0-9]*[-~f.]*[0-9]*\) -> \\dok{\1}
%%%%%%%%% Stellenangaben bis hierher %%%%%%%%%%%%%%
\makeatother
