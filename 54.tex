\chapter{Brief des Georg von Laodicaea an Macedonius, Basilius, Cecropius und Eugenius}
\label{ch:BriefGeorg}
\begin{praefatio}
  \begin{description}
  \item[Anfang 358]Der Brief beklagt die Aufnahme des A"etius\enindex{A"etius!Diakon in Antiochien}
  in den Klerus von Antiochien durch den dortigen Bischof Eudoxius;\enindex{Eudoxius!Bischof von Antiochien/Konstantinopel} 
  zur weiteren Entwicklung s. Dok. \ref{ch:Ankyra358} 
  und Dok. \ref{ch:Sirmium358}. \\
  Nach dem Tod des Bischofs Leontius von Antiochien\enindex{Leontius!Bischof von Antiochien} konnte
    Eudoxius,\enindex{Eudoxius!Bischof von Antiochien/Konstantinopel} bisher Bischof von Germanicia in 
    Armenien (\sozhe[i]{III 14,42}; \thhe[i]{II 25,1};
    \athha[i]{4,2}; vgl. Dok. \ref{sec:Sirmium351:Teilnehmer}, Nr. 4), Teilnehmer der 
    Synode von Antiochien\esindex{Antiochien!a. 341}
    (vgl. \dok[i]{41.2,2,5}), der �stlichen Teilsynode von Serdica\esindex{Serdica!a. 343}
    (\dok[i]{43.13}, Nr. 19) und einer der 345 nach Mailand\esindex{Mailand!a. 345} Delegierten
    (vgl. \dok[i]{44}, Einleitung), in einer umstrittenen Wahl dessen
    Nachfolger werden (\athsyn[i]{12,2}; \socrhe[i]{II 37,7--9}; \sozhe[i]{IV 12,3~f.}).  Er
    suchte offenbar Unterst�tzung durch Ursacius von Singidunum,\enindex{Ursacius!Bischof von Singidunum}
    Valens von Mursa\enindex{Valens!Bischof von Mursa} und 
    Germinius von Sirmium\enindex{Germinius!Bischof von Sirmium} aus dem Westen, so da�
    er auf einer antiochenischen Synode\esindex{Antiochien!a. 357} zusammen mit Acacius von
    Caesarea\enindex{Acacius!Bischof von Caesarea} (vgl. Dok. \ref{sec:Sirmium351:Teilnehmer}, Nr. 18 und
    Dok. \ref{sec:ArlesMailand:Teilnehmer}, Nr. 22; er wird auf den Synoden von Seleucia 359\esindex{Seleucia!a. 359} 
    und Konstantinopel 360\esindex{Konstantinopel!a. 360} im Zentrum stehen)
    und Uranius von Tyrus\enindex{Uranius!Bischof von Tyrus} (vgl. Dok. \ref{ch:Seleucia359}; 
    Dok. \ref{sec:Seleucia359:Bekenntnis},7, Nr. 3)
    die �2. sirmische Formel�\esindex{Sirmium!a. 357}
    (Dok. \ref{ch:Sirmium357:2Formel}) �bernahm (\sozhe[i]{IV 12,5--7}). Ferner pflegte er den Kontakt zu den beiden �Anhom�ern�\enindex{Anhom�er}
    A"etius,\enindex{A"etius!Diakon in Antiochien} der von Eudoxius'\enindex{Eudoxius!Bischof von Antiochien/Konstantinopel} Vorg�nger 
    Leontius\enindex{Leontius!Bischof von Antiochien} zum Diakon in
    Antiochien\enindex{Antiochien} geweiht worden war (\sozhe[i]{IV 12,1}; zu A"etius
    vgl. Dok. \ref{ch:Aetius}), und Eunomius,\enindex{Eunomius!Bischof von Cyzicus} 
    den Eudoxius\enindex{Eudoxius!Bischof von Antiochien/Konstantinopel} selbst zum
    Diakon in Antiochien weihte (\philost[i]{IV 5}; zu Eunomius
    vgl. Dok. \ref{sec:Kpl360:Verurteilung}, Einleitung).  Kaiser
    Constantius\enindex{Constantius!Kaiser} verbannte Eudoxius\enindex{Eudoxius!Bischof von Antiochien/Konstantinopel} 
    nach massiven Protesten, wie sie
    auch aus diesem Brief erkennbar werden, noch im Jahr 358 in seine
    Heimat Armenien
    (vgl. Dok. \ref{sec:Sirmium358:BriefConstantius}), von wo der aber noch vor der Synode von Seleucia\esindex{Seleucia!a. 359} zur�ckkehren konnte
    (vgl. Dok. \ref{ch:Seleucia359}). Nach der Ver�nderung der
    kirchenpolitischen Lage nach der Synode von
    Konstantinopel\esindex{Konstantinopel!a. 359} wurde er 360 Bischof
    der Hauptstadt\enindex{Konstantinopel} anstelle des abgesetzten
    Macedonius\enindex{Macedonius!Bischof von Konstantinopel} (\socrhe[i]{II 43,7--16};
    \sozhe[i]{IV 26,1}; Dok. \ref{sec:Kpl360:Teilnehmer}) und starb im Jahr 370
    (\socrhe[i]{IV 14,2}). \\  
    Der Verfasser dieses Briefes ist Georg von Laodicaea\enindex{Georg!Bischof von Laodicaea} (\sozhe IV
    13,1; vgl. \dok{41.2,2,5} Anmerkung), der neben Basilius von Ancyra\enindex{Basilius!Bischof von Ancyra}
    einer der wichtigsten Vertreter der sogenannten �Hom�usianer�\enindex{Hom�usianer} war
    (vgl. Dok. \ref{ch:GeorgBasilius359}), dessen Spur sich aber nach der Synode von Seleucia\esindex{Seleucia!a. 359}
    359/360 verliert (\thhe[i]{II 31,6~f.} ist wohl in Georg von
    Alexandrien\enindex{Georg!Bischof von Alexandrien} zu
    korrigieren; vgl. Dok. \ref{ch:SynAntiochien360}, Einleitung.). Aller Wahrscheinlichkeit nach ist er bald nach der
    Synode\esindex{Seleucia!a. 359} verstorben. Sein von
    Acacius von Caesarea\enindex{Acacius!Bischof von Caesarea} (laut \philost[i]{V 1} noch in Konstantinopel 360)
    eingesetzter Nachfolger war Pelagius\enindex{Pelagius!Bischof von Laodicaea} (vgl. \socrhe[i]{III 25,18}).
    Ebenfalls ca. 360 wurde Apolinaris\enindex{Apolinaris!Bischof von Laodicaea} von den Homousianern\enindex{Homousianer} zum Bischof eingesetzt; vgl. Dok. \ref{sec:SynAlex362:Unterschriften}.
  \item[�berlieferung]Dieser Brief ist nur bei Sozomenus �berliefert,
    der ihn wahrscheinlich der Synagoge des Sabinus\enindex{Sabinus!Bischof von Heraclea} entnommen
    hat. Eventuell handelt es sich bei dem Brief nur um ein
    Fragment. Urspr�nglich war der Brief auch im Anhang des Briefes
    der Synode von Ancyra\esindex{Ancyra!a. 358} (Dok. \ref{ch:Ankyra358}) �berliefert
    (vgl. Dok. \ref{ch:Ankyra358},4), ist dort heute allerdings verloren.
  \item[Fundstelle]\sozhe{IV 13,2--3}
    (\cite[155,21--156,4]{Hansen:Soz}).
  \end{description}
\end{praefatio}
\begin{pairs}
\selectlanguage{polutonikogreek}
\begin{Leftside}
\beginnumbering
\specialindex{quellen}{chapter}{Sozomenus!h.\,e.!IV 13,2--3}
\pstart
\bezeugung{BC=b T}
\kap{1}Kur'ioic \var{timiwt'atoic}{mou <agiwt'atoic \sg{T}}
Makedon'iw|,\nindex{Macedonius!Bischof von Konstantinopel} Basile'iw|,\nindex{Basilius!Bischof von Ancyra}
 \var{Kekrop'iw|}{ka`i krop'iw ka`i \sg{T}},\nindex{Cecropius!Bischof von Nikomedien} 
E>ugen'iw|\nindex{Eugenius!Bischof von Nicaea} Ge'wrgioc\nindex{Georg!Bischof von Laodicaea} \varom{>en kur'iw| qa'irein}{T}.
\pend
\pstart
\kap{2}t`o >Aet'iou\nindex{A"etius!Diakon in Antiochien} nau'agion sqed'on pou p~asan kate'ilhfe t`hn
>Antioq'ewn.\nindex{Antiochien} to`uc g`ar par'' <um~in
\var{>atimazom'enouc}{a>itiazom'enouc \sg{T}} majht`ac to~u duswn'umou >Aet'iou\nindex{A"etius!Diakon in Antiochien} p'antac katalab`wn E>ud'oxioc\nindex{Eudoxius!Bischof von Antiochien/Konstantinopel} e>ic klhriko`uc prob'alletai, >en to~ic m'alista tetimhm'enoic >'eqwn t`on a<iretik`on >A'etion.\nindex{A"etius!Diakon in Antiochien} katal'abete \varom{o>~un}{T} t`hn thlika'uthn p'olin, m`h t~w| nauag'iw| a>ut~hc ka`i <h o>ikoum'enh parasur~h|. 
 ka`i e>ic ta>ut`on gen'omenoi, <'osouc ka`i gen'esjai
>egqwre~i, par`a t~wn >'allwn >episk'opwn <upograf`ac >apait'hsate,
<'ina ka`i >A'etion\nindex{A"etius!Diakon in Antiochien} >ekb'alh| t~hc >Antioq'ewn\nindex{Antiochien} \varom{>ekklhs'iac}{T}
E>ud'oxioc\nindex{Eudoxius!Bischof von Antiochien/Konstantinopel} ka`i to`uc a>uto~u majht`ac \varadd{>'ontac}{ka`i \sg{T}},
proqeirisj'entac e>ic \var{kan'ona}{kan'onac \sg{b}},
\var{>ekk'oyh|}{>ekp'emyh \sg{B}}. >`h >e`an
\var{>epime'inh|}{>epim'enh \sg{T}} met`a >Aet'iou\nindex{A"etius!Diakon in Antiochien} >an'omoion kal~wn
ka`i to`uc to~uto tolm~wntac l'egein t~wn m`h leg'ontwn protim~wn,
o>'iqetai \var{<hm~in}{<um~in \sg{b}}, <wc fj'asac >'efhn, t'ewc <h >Antioq'ewn.\nindex{Antiochien}
\pend
\endnumbering
\end{Leftside}
\begin{Rightside}
\begin{translatio}
\beginnumbering
\pstart
Georg gr��t im Herrn die hochverehrten Herren Macedonius\hist{Macedonius, Gegenbischof, dann Nachfolger
  des Paulus von Konstantinopel,\enindex{Paulus!Bischof von Konstantinopel} Teilnehmer der Synode von Seleucia
  359\esindex{Seleucia!a. 359} (Dok. \ref{sec:Seleucia359:Protokolle},3), 360 auf der Synode von
  Konstantinopel\esindex{Konstantinopel!a. 360} abgesetzt (s. Dok. \ref{sec:Kpl360:Teilnehmer}, Einleitung) 
  und wohl bald danach verstorben (\sozhe[i]{IV 26,1}).},
Basilius\hist{Basilius von Ancyra, vgl. \dok{40.3,1}; \ref{sec:Sirmium351:Photin}
  und \ref{ch:Ankyra358}.},
Cecropius\hist{Cecropius von Nikomedien, Nachfolger des
  Amphion\enindex{Amphion!Bischof von Nikomedien}, vgl. \dok{43.11,1}, vgl. ferner
  Dok. \ref{sec:Sirmium351:Teilnehmer} Nr. 6 mit Anm.} und Eugenius\hist{Eugenius von Nicaea,
  Teilnehmer einer antiochenischen Synode\esindex{Antiochien!a. 352} (aller Wahrscheinlichkeit
  nach 352, s. Dok. \ref{ch:ArlesMailand}, Einleitung), 
  die Athanasius\enindex{Athanasius!Bischof von Alexandrien} absetzte (\sozhe{IV 8,4}), eventuell Unterzeichner in
  Konstantinopel 360\esindex{Konstantinopel!a. 360} (vgl. \ref{sec:Kpl360:Teilnehmer}, Nr. 14), starb 370
  (\philost IX 8).}.
\pend
\pstart
Der Schiffbruch des A"etius hat beinahe ganz Antiochien in Bedr�ngnis
gebracht. Denn die von Euch entehrten Sch�ler des unseligen A"etius hat
Eudoxius allesamt an sich gezogen und im Klerus untergebracht, wobei der
H�retiker A"etius unter den meist Geehrten war. Helft also dieser
Stadt, damit nicht durch ihren Schiffbruch der ganze Erdkreis
untergeht! Versammelt Euch, soweit es m�glich ist, an einem Ort, und
erbittet die Unterschriften von den �brigen Bisch�fen, damit Eudoxius
A"etius aus der Kirche Antiochiens ausschlie�t und seine Sch�ler, die
in die Klerikerliste eingetragen wurden, wieder entfernt! Oder falls er
dabei bleibt, mit A"etius �ungleich� zu
rufen und diejenigen, die dies zu sagen wagen, denen vorzieht, die dies
nicht sagen, dann ist die Kirche Antiochiens, wie ich vorhin bereits sagte, f�r uns gestorben.
\pend
\endnumbering
\end{translatio}
\end{Rightside}
\Columns
\end{pairs}
\selectlanguage{german}

%%% Local Variables: 
%%% mode: latex
%%% TeX-master: "dokumente"
%%% coding: latin-1
%%% End: 
