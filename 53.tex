\chapter{Korrespondenz des Potamius von Lissabon}
\label{ch:BriefPotamius}
\begin{praefatio}
  \begin{description}
  \item[um 357?]Der Satz des ersten Dokuments ist nach Phoebadius
    von Agen\enindex{Phoebadius!Bischof von Agen} ein Zitat aus einem Brief des 
    Potamius von Lissabon,\enindex{Potamius!Bischof von Lissabon} den er
    im Zusammenhang mit der Synode von Sirmium 357\esindex{Sirmium!a. 357}
    (vgl. Dok. \ref{ch:Sirmium357:2Formel}) in den Osten 
    und Westen verschickt hatte. Die Einreihung des
    Potamius\enindex{Potamius!Bischof von Lissabon} unter die �Arianer�\enindex{Arianer} 
    bedeutet hier, da� er die theologische Erkl�rung 
    dieser Synode bef�rwortet hat. Laut \hilsyn[i]{3} ist
    er an ihrer Abfassung beteiligt gewesen
    (vgl. aber \hilconst[i]{23}; \hilconsto[i]{26}, wo Valens von Mursa\enindex{Valens!Bischof von Mursa} und
    Ursacius von Singidunum\enindex{Ursacius!Bischof von Singidunum} als Hauptverantwortliche genannt werden). Seine
    theologische Position bleibt unklar, da nur diese Aussage zur
    Inkarnation (vgl. auch Dok. \ref{ch:Sirmium357:2Formel},6) 
    �berliefert ist (vgl. auch \libprec[i]{32} und \hilcoll[i]{B III 2}) und fr�here Aussagen nicht bekannt sind. 
    Unter Umst�nden handelt es sich um eine (bewu�t?) verzerrende Wiedergabe
    von Dok. \ref{ch:Sirmium357:2Formel},6. \\
    Das zweite Dokument gibt einen hinsichtlich seiner Authentizit�t
    nicht ganz sicheren Ausschnitt aus einem Brief des
    Athanasius von Alexandrien\enindex{Athanasius!Bischof von Alexandrien} an 
    Potamius\enindex{Potamius!Bischof von Lissabon} wieder, in dem er die
    Beschreibung des Sohnes Gottes als Gesch�pf
    kritisiert. Bekanntlich ist diese Position von den Hom�ern nie
    vertreten worden. Insofern geh�rt dieses Brieffragment in die auch
    sonst h�ufig bei Athanasius\enindex{Athanasius!Bischof von Alexandrien} zu beobachtende Strategie, den Hom�ern
    Aussagen des Arius\enindex{Arius!Presbyter in Alexandrien} zu unterstellen. Der Brief
    k�nnte sowohl vor als auch nach der Synode von Sirmium 357\esindex{Sirmium!a. 357} zu
    datieren sein. \\
    Der Kontakt zwischen Potamius\enindex{Potamius!Bischof von Lissabon} 
    und Athanasius\enindex{Athanasius!Bischof von Alexandrien}
    scheint fortbestanden zu haben, da Potamius\enindex{Potamius!Bischof von Lissabon} nach der Synode von
    Rimini\esindex{Rimini!a. 359} und nach dem Tod des Kaisers Constantius\enindex{Constantius!Kaiser} einen Brief an
    Athanasius\enindex{Athanasius!Bischof von Alexandrien} geschickt hat, in dem er sich als eifriger Bef�rworter der
    nicaenischen Terminologie erweist
    (s. Dok. \ref{ch:BriefPotamiusAthanasius} und auch den Traktat
    \textit{De substantia} (CPL 544),\exindex{Potamius!De
    substantia} falls er authentisch ist).
  \item[�berlieferung]Phoebadius zitiert in seiner Schrift \textit{Contra
    Arianos}, die nur in einer Handschrift (Cod. Vossius
    lat. F 58 [V]) �berliefert ist, einen kurzen Auszug aus einem Rundbrief des
    Potamius\enindex{Potamius!Bischof von Lissabon}. Ebenfalls um einen Auszug handelt es sich bei dem Brieffragment
    des Athanasius\enindex{Athanasius!Bischof von Alexandrien} an Potamius,\enindex{Potamius!Bischof von Lissabon} den Alcuin (offensichtlich in einer
    �bersetzung aus dem Griechischen) in seinem \textit{Liber contra haeresim
    Felicis} zitiert, der ebenfalls nur in einer Handschrift
    (Cod. Vaticanus Palatinus lat. 290 [P]) �berliefert ist.
  \item[Fundstelle]\refpassage{Phoeb1}{Phoeb2} Phoeb., c.\,Ar. 5,1
    (\cite[27]{PhoebadiusAginnensis:1985});
    \refpassage{Alcuin1}{Alcuin2} Alcuin, Liber contra haeresim
    Felicis 61 (\cite[90,7--18]{Blumenshine:1980}).
  \end{description}
\end{praefatio}
\autor{Auszug aus einem Rundbrief des Potamius?}
\begin{pairs}
  \selectlanguage{latin}
  \begin{Leftside}
    \beginnumbering
    \specialindex{quellen}{chapter}{Phoebadius!c.\,Ar.!5,1}\specialindex{quellen}{chapter}{Alcuin!haer.\,Fel.!61}
    \pstart
    \bezeugung{V}\kap{1,1}\footnotesize{Et\edlabel{Phoeb1} idcirco duplicem hunc statum non
     \varabb{coniunctum}{\sglat{coni. Demeulenaere} coniuctum \sglat{V}} sed confusum vultis videri, ut
     etiam \varlat{unius}{unus}{coni. Migne} vestrum, \varabb{id est epistola \varlat{Potamii}{Potami}{coni. Beza}}{glossam \sglat{susp. Barthius}},\nindex{Potamius!Bischof von Lissabon} quae \varlat{ad Orientem et Occidentem}{ab Oriente et Occidente}{coni. Migne} transmissa est, qua
      adserit}
    \pend
    \pstart
   \kap{2}carne et spiritu Christi 
    \varabb{coagulatis}{\sglat{V\corr} coagolatis \sglat{V*}} per
    sanguinem Mariae et in unum corpus redactis passibilem deum
    factum.\edlabel{Phoeb2}
    \pend
    % \endnumbering
  \end{Leftside}
  \begin{Rightside}
    \begin{translatio}
      \beginnumbering
      \pstart
      \footnotesize{Und deshalb wollt Ihr, da� dieser zweifache
       Stand nicht verbunden, sondern vermischt erscheint, wie es
        auch einer von Euch will, das hei�t der Brief des Potamius,
        der in den Osten und in den Westen geschickt worden ist. In
        ihm erkl�rt er:}
      \pend
      \pstart
      Dadurch, da� Fleisch und Geist Christi durch das Blut Marias
      geronnen waren und zu einem einzigen K�rper gemacht worden
      waren, ist Gott leidensf�hig geworden.
      \pend
      \endnumbering
    \end{translatio}
  \end{Rightside}
  \Columns
\end{pairs}
\selectlanguage{german}

\autor{Auszug aus einem Brief des Athanasius von Alexandrien an Potamius}
\begin{pairs}
  \selectlanguage{latin}
  \begin{Leftside}
    %\beginnumbering
    \pstart
    \bezeugung{P}\kap{2,1}\footnotesize{Item\edlabel{Alcuin1} beatus Athanasius\nindex{Athanasius!Bischof von Alexandrien} in
     epistola ad Potamium\nindex{Potamius!Bischof von Lissabon} episcopum inter cetera:}
    \pend
    \pstart
    \kap{2}quaero itaque abs te quisquis huius es scrutator archani, quomodo
    deus ex nihilo cuncta condiderit, quomodo \bibelcf{una hominis costa
    integrum hominem fecerit}{Gen 2,21~f.},\bindex{Genesis!2,21~f.} qualiter
    \bibelcf{\varabb{et Aegyptiorum aquas}{\sglat{coni. Foggini} ex
        (\sglat{del. Blumenshine})
        Egyptiorum aliquas \sglat{P}} in sanguine
    verterit}{Ex 7,20},\bindex{Exodus!7,20} quomodo \bibelcf{de durissima petra liquida fluenta 
    produxerit}{Ex 17,6},\bindex{Exodus!17,6}
    qua ratione ille ipse quem creaturam putas caro sit factus ex
    virgine? quomodo potuit \bibelcf{de quinque panibus quinque milia hominum
    saturare}{Mt 14,21},\bindex{Matthaeus!14,21} vel \bibelcf{ad discipulos cum corpore clausis foribus 
    introire}{Io 20,26}?\bindex{Johannes!20,26}
    aut si fateris \varabb{et}{\sglat{coni. Erl} ex \sglat{del. Blumenshine}} intelligentiae tuae vires excedere et
    credenda esse potius quam explicanda concedis, quid inveniri
    iniquius potest quam simpliciter me credere nole quod lego et
    praesumptive credere vel quod non lego?\edlabel{Alcuin2}
    \pend
    \endnumbering
  \end{Leftside}
  \begin{Rightside}
    \begin{translatio}
      \beginnumbering
      \pstart
      \footnotesize{Ebenso schrieb der selige Athanasius in einen Brief an
       Bischof Potamius unter anderem folgendes:}
      \pend
      \pstart
      Daher erfrage ich von Dir als Erforscher des Verborgenen, auf
      welche Weise Gott das All aus nichts schuf, auf welche Weise er
      einen vollst�ndigen Menschen aus einer Rippe eines Menschen
      machte, wie er sogar die Wasser der �gypter in Blut verwandelte,
      wie er aus v�llig hartem Felsen flie�endes Wasser hervorbrachte,
      auf welche Weise jener, den du f�r ein Gesch�pf h�ltst, selbst
      Fleisch wurde aus der Jungfrau? Wie konnte er mit f�nf Broten
      f�nftausend Menschen s�ttigen, oder mit seinem K�rper zu den
      J�ngern durch verschlossene T�ren hindurchgehen? Aber wenn du
      eingestehst, was sogar die Kr�fte deines Verstandes �bersteigt, und
      zugestehst, da� Glauben besser als Erkl�ren ist, was kann
      Schlechteres erdacht werden als einfach nicht glauben zu wollen,
      was ich sage, und von vorneherein glauben zu wollen, was ich
      nicht sage?
      \pend
      \endnumbering
    \end{translatio}
  \end{Rightside}
  \Columns
\end{pairs}
\selectlanguage{german}
%%% Local Variables: 
%%% mode: latex
%%% TeX-master: "dokumente"
%%% End: 
