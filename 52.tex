\chapter{Nachricht �ber eine gallische Synode im Jahr 358}
\label{ch:Sirmium357:Ablehnung}
\begin{praefatio}
  \begin{description}
  \item[358]Nach \hilsyn[i]{28} fand diese gallische Synode\esindex{Gallien!a. 358} zeitgleich
   mit einer Synode in Ancyra\esindex{Ancyra!a. 358} (vgl. Dok. \ref{ch:Ankyra358}) 
   statt, auf der ebenfalls die �2. sirmische Formel� abgelehnt wurde, also um Ostern
    358. Wohl aufgrund der literarischen T�tigkeit des Hilarius von
    Poitiers\enindex{Hilarius!Bischof von Poitiers} (\textit{Liber 1
      adversus Valentem et Ursacium})\exindex{Hilarius!adv. Val. et
      Urs.!I} und des
    Phoebadius von Agen\enindex{Phoebadius!Bischof von Agen}
    (\textit{Contra Arianos})\exindex{Phoebadius!c. Ar.} entwickelte sich in
    Gallien\enindex{Gallien} schon bald nach der sirmischen Synode von 357\esindex{Sirmium!a. 357} (Dok. \ref{ch:Sirmium357:2Formel})
    eine Opposition gegen die dort formulierte theologische Erkl�rung;
    auch das vom Verbot der Usia-Terminologie betroffene Nicaenum\esindex{Nicaea!a. 325}
    (\dok[i]{26}) wurde verteidigt. Die Schrift
    \textit{Contra Arianos} des Phoebadius\enindex{Phoebadius!Bischof von Agen} ist aller
    Wahrscheinlichkeit nach vor der gallischen Ostersynode\esindex{Gallien!a. 358} verfa�t
    worden, da die Verurteilung der �2. sirmischen Formel� durch
    eine gallische Synode nicht erw�hnt wird.
  \item[�berlieferung]Diese gallische Synode l��t sich nur indirekt aus diesen
  Bemerkungen bei Hilarius in seiner Schrift \textit{De synodis} erschlie�en.
  Entsprechende Schriftst�cke der gallischen Synode an den im Exil
    befindlichen Hilarius\enindex{Hilarius!Bischof von Poitiers} sind nicht �berliefert. 
  \item[Fundstelle]\hilsyn{2} (\cite[481 A]{Hil:Syn}).
  \end{description}
\end{praefatio}
\begin{pairs}
\selectlanguage{latin}
\begin{Leftside}
\beginnumbering
\specialindex{quellen}{chapter}{Hilarius!syn.!2}
\pstart
\bezeugung{Hil.}\kap{}Sed beatae fidei vestrae litteris sumptis, quarum lentitudinem ac raritatem 
de exilii mei et longitudine et secreto intellego constitisse, gratulatus sum in
 domino incontaminatos vos et inlaesos ab omni contagio detestandae hereseos 
perstitisse vosque conparticipes exilii mei, in quod me Saturninus\nindex{Saturninus!Bischof von Arles} ipsam 
conscientiam suam veritus circumvento imperatore\nindex{Constantius!Kaiser} detruserat, negata ipsi usque 
hoc tempus toto iam triennio communione fide mihi ac spiritu cohaerere et missam
 proxime vobis ex Sirmiensi\sindex{Sirmium!a. 357} oppido infidelis fidei impietatem non modo non 
suscepisse, sed nuntiatam etiam significatamque damnasse. 
\pend
\endnumbering
\end{Leftside}
\begin{Rightside}
\begin{translatio}
\beginnumbering
\pstart
Aber nachdem ich Briefe, die eueren seligen Glauben bezeugen, erhalten
hatte~-- ich habe infolge der weit entfernten Abgeschiedenheit meines
Exils Verst�ndnis daf�r, da� sich nichts daran �ndern l��t, da� diese
Briefe nur langsam und selten eintreffen~--, war ich dankbar im Herrn, da� ihr unbefleckt
und von jeder Einwirkung der verdammenswerten H�resie unversehrt ausgeharrt habt und als Teilhaber an meiner
Verbannung, in die mich Saturninus\hist{Bischof von Arles,
  vielleicht schon zur Zeit der Synode von Serdica\esindex{Serdica!a. 343}
  (vgl. \dok[i]{43.3}, Nr. 109), wobei die Identit�t mit dem dort
  genannten Bischof Saturninus unsicher ist. Seine zu vermutende Anwesenheit auf der 
  Synode von Arles\esindex{Arles!a. 353} ist nicht bezeugt (vgl. Dok. \ref{ch:ArlesMailand}), 
  als Teilnehmer der Synode von Mailand 355\esindex{Mailand!a. 355}
  ist er aber belegt (Dok. \ref{sec:ArlesMailand:Teilnehmer}, Nr. 4).
  Auf der Synode von B�ziers 356,\esindex{B�ziers!a. 356} auf der Hilarius als Bischof von
  Poitiers\enindex{Hilarius!Bischof von Poitiers} abgesetzt wurde (\virill[i]{100}; \hilsyn[i]{2}; \hilconsta[i]{2}; 
  Dok. \ref{sec:Kpl360:HilConst},2), mu� er eine f�hrende Rolle
  gespielt haben. Nach dieser Notiz
  bei Hilarius\enindex{Hilarius!Bischof von Poitiers} hatten gallische Bisch�fe inzwischen sowohl
  die theologische Erkl�rung von Sirmium\esindex{Sirmium!a. 357} als h�retisch verurteilt,
  Saturnin von Arles exkommuniziert und an der Gemeinschaft mit dem abgesetzten und ins Exil nach Phrygien vertriebenen Hilarius von Poitiers festgehalten (vgl. Dok. \ref{ch:SynParis},4). Nach
  Dok. \ref{sec:Kpl360:HilConst},2 mu� Saturninus sich Anfang 360 in Konstantinopel\enindex{Konstantinopel}
  aufgehalten haben (als Mitglied der Delegation aus Rimini?\esindex{Rimini!a. 359}). Offenbar hat eine gallische
  Synode 360/61 (Dok. \ref{ch:SynParis},4)\esindex{Gallien!a. 360/61} ihn noch einmal exkommuniziert und abgesetzt. Weitere Nachrichten �ber ihn sind nicht �berliefert.} aus Furcht
vor seinem eigenen Gewissen unter T�uschung des Kaisers
getrieben hatte, mit mir eins im Glauben und im Geiste wart,
w�hrend man mir doch bis zu dieser Zeit schon ganze drei Jahre
die Gemeinschaft verweigerte,\hist{D.\,h. seit seiner
  Absetzung auf der Synode von B�ziers\esindex{B�ziers!a. 356} und anschlie�enden Verbannung
  nach Phrygien im Jahre 356.} und da� ihr die Frevelhaftigkeit der ungl�ubigen Glaubenserkl�rung, die euch vor kurzem aus Sirmium geschickt worden war, nicht nur nicht akzeptiert, sondern auch gemeldet, angezeigt und somit verurteilt habt. 
\pend
\endnumbering
\end{translatio}
\end{Rightside}
\Columns
\end{pairs}
\selectlanguage{german}

%%% Local Variables: 
%%% mode: latex
%%% TeX-master: "dokumente"
%%% coding: latin-1
%%% End: 
