\section{Fragmente eines Briefes des A"etius}
\label{sec:Aetius-Brief}
\begin{praefatio}
  \begin{description}
  \item[vor Dezember 359]Neben Dok. \ref{sec:Aetius-Syntagmation} sind Fragmente eines weiteren Briefes des A"etius\enindex{A"etius!Diakon in Antiochien} erhalten, dessen Entstehungskontext im Einzelnen unklar ist. Theodoret zufolge wurde aus ihm w�hrend der Verhandlungen in Konstantinopel im Dezember 359\esindex{Konstantinopel!a. 359} zitiert (abgesehen von einem leichten �berschu� ist der Text dieses Dokuments identisch mit Dok.
    \ref{sec:Kpl360:Verurteilung},1,4). Sollte diese Zuordnung richtig
    sein, erg�be sich daraus eine Datierung vor Dezember 359. Der
    Inhalt der Fragmente f�gt sich zum sog. Syntagmation des Aetius\enindex{A"etius!Diakon in Antiochien}
    (Dok. \ref{sec:Aetius-Syntagmation}). Im Unterschied zu diesem ist
    aber keine feste Thesenform erkennbar, sondern es ist eher ein
    fortlaufend argumentierender Text zu vermuten. Im Hintergrund des
    erhaltenen Textes steht eine Diskussion um das Verst�ndnis von
    1Cor 8,6\ebindex{Korinther I!8,6} (vgl. auch Eunom., apol. 5; 25~f.).\exindex{Eunomius!apol.!5}\exindex{Eunomius!apol.!25~f.} Vielleicht geh�ren diese Fragmente zu einem Brief aus der Kampagne, die \philost
IV 11 im Rahmen der Synode von Seleucia\esindex{Seleucia!a. 359} erw�hnt; vgl. Dok. \ref{ch:Kpl360}, Einleitung.
  \item[�berlieferung]Der Brief ist nur fragmentarisch in Gestalt der bei Basilius, spir. 2,4, gebotenen Zitate �berliefert. Das zweite und dritte Fragment begegnet daneben in Theodorets Darstellung (= Dok. \ref{sec:Kpl360:Verurteilung},1,4) der Synode von Konstantinopel,\esindex{Konstantinopel!a. 359} auf der ihm bzw. seiner Quelle zufolge Eustathius von Sebaste\enindex{Eustathius!Bischof von Sebaste} aus einer \griech{>'ekjesis} des A"etius\enindex{A"etius!Diakon in Antiochien} diese Stellen vorlas (Dok. \ref{sec:Kpl360:Verurteilung},1,3). Im Unterschied zu Theodoret gibt Basilius\enindex{Basilius!Bischof von Caesarea} das zweite Fragment (ebenso wie das erste) in indirekter Rede wieder, weshalb Theodorets Textform bez�glich des Pr�dikates hier die originale(re) ist (und man eine analoge Originalform f�r das Pr�dikat im ersten Fragment annehmen kann); andererseits enth�lt das dritte Fragment bei Basilius einen Satz mehr, den Theodoret bzw. seine Quelle wohl nicht hatte. Wie aus Basilius' einleitender Bemerkung (vgl. � 1) hervorgeht, existierte offenbar eine Sammlung von A"etiusbriefen, aus der (oder einem Florileg?) vermutlich sowohl Basilius als auch Eustathius sch�pften; die insgesamt umfangreichere Darstellung Theodorets l�sst sich dagegen nicht allein aus dessen etwaiger Benutzung des Basilius erkl�ren. \\
 Eine kritische Edition von \textit{De spiritu sancto} des Basilius liegt bisher nicht vor. Die Ausgabe von \cite{Basilius:1968} bietet nur  eine Auswahl von Handschriften, wodurch im Prinzip ein Nachdruck
    von PG 32 erreicht wird (vgl. \cite[238]{Basilius:1968}). F�r � 2~f. ist au�erdem Theodoret (= Dok.
    \ref{sec:Kpl360:Verurteilung},1,4) zu vergleichen; relevante
    Varianten des Theodoret-Textes sind hier in den textkritischen Apparat aufgenommen. 
  \item[Fundstelle]Bas., spir. 2,4 (\cite[260,8--18]{Basilius:1968}).
  \end{description}
\end{praefatio}
\begin{pairs}
\selectlanguage{greek}
\begin{Leftside}
  %\beginnumbering
  \specialindex{quellen}{section}{Basilius!spir.!2,4}
  \pstart
  \kap{1}\footnotesize{>'Esti g'ar ti a>uto~ic
    palai`on s'ofisma <up`o >Aet'iou\nindex{A"etius!Diakon in Antiochien} to~u prost'atou t~hc a<ir'esewc
    ta'uthc >exeurej'en, <`oc >'egray'e pou t~wn <eauto~u >epistol~wn,
    l'egwn;}
  \pend
  \pstart
  t`a >an'omoia kat`a t`hn f'usin >anomo'iwc prof'eresjai;
  \pend
  \pstart
  \kap{2}\footnotesize{ka`i >an'apalin;}
  \pend
  \pstart
  \bezeugungpartlang{t`a >anomo'iwc profer'omena >an'omoia e>~inai kat`a
    t`hn \var{f'usin}{o>us'ian \sg{Thdt.}}.}{t`a \dots\ f'usin}{vgl. Dok. \ref{sec:Kpl360:Verurteilung},1,4
    (\edpageref{sec:Kpl360:Verurteilung-1-4-1},\lineref{sec:Kpl360:Verurteilung-1-4-1}--\edpageref{sec:Kpl360:Verurteilung-1-4-2},\lineref{sec:Kpl360:Verurteilung-1-4-2})}
  \pend
  \pstart
  \kap{3}\footnotesize{ka`i e>ic martur'ian to~u l'ogou t`on
    >ap'ostolon >epesp'asato l'egonta;}
  \pend
  \pstart
  \bezeugungpartlang{((\bibel{e<~ic je`oc \var{ka`i}{<o \sg{Thdt.}} pat'hr, >ex o<~u t`a p'anta; ka`i e<~ic
    k'urioc >Ihso~uc Qrist'oc, di'' o<~u t`a p'anta.}{1Cor
    8,6}\bindex{Korinther I!8,6}))}{e<~ic \dots\ p'anta}{vgl. Dok. \ref{sec:Kpl360:Verurteilung},1,4
    (\edpageref{sec:Kpl360:Verurteilung-1-4-3},\lineref{sec:Kpl360:Verurteilung-1-4-3}--\edpageref{sec:Kpl360:Verurteilung-1-4-4},\lineref{sec:Kpl360:Verurteilung-1-4-4})} <wc o>~un >'eqousin a<i
  fwna`i pr`oc >all'hlac, o<'utwc <'exousi, {\footnotesize{fhs'i}}, ka`i a<i
  di'' a>ut~wn shmain'omenai f'useic; \bezeugungpartlang{>an'omoion d`e t~w| >ex o<~u t`o
  di'' o<~u; >an'omoioc >'ara ka`i t~w| patr`i <o u<i'oc.}{>an'omoion
  \dots\ u<i'oc}{vgl. Dok. \ref{sec:Kpl360:Verurteilung},1,4
    (\edpageref{sec:Kpl360:Verurteilung-1-4-5},\lineref{sec:Kpl360:Verurteilung-1-4-5}--\edpageref{sec:Kpl360:Verurteilung-1-4-6},\lineref{sec:Kpl360:Verurteilung-1-4-6})}
  \pend
  \endnumbering
\end{Leftside}
\begin{Rightside}
  \begin{translatio}
    \beginnumbering
    \pstart
    \footnotesize{Es gibt n�mlich bei ihnen einen alten Trugschlu�, der von A"etius,
    dem Anf�hrer dieser H�resie erfunden wurde, der irgendwo in seinen
    eigenen Briefen mit folgenden Worten schreibt:}
    \pend
    \pstart
    Die Dinge, die ungleich im Wesen seien, w�rden auf ungleiche Weise
    hervorgebracht;
    \pend
    \pstart
    \footnotesize{und wiederum:}
    \pend
    \pstart
    Die Dinge, die auf ungleiche Weise hervorgebracht w�rden, seien
    ungleich im Wesen.
    \pend
    \pstart
    \footnotesize{Und zum Beweis seiner Rede zog er den Apostel heran, der sagt:}
    \pend
    \pstart
    �Ein Gott und Vater, aus dem alles ist; und ein Herr Jesus
    Christus, durch den alles ist.� Wie sich also diese S�tze zu
    einander verhalten, so, sagt er, werden sich auch die beiden
    Wesensarten, die durch sie aufgezeigt werden, verhalten; ungleich
    aber ist �durch was� dem �aus was�; folglich ist auch der Vater
    dem Sohn ungleich.
    \pend
    \endnumbering
  \end{translatio}
\end{Rightside}
\Columns
\end{pairs}
\selectlanguage{german}

%%% Local Variables: 
%%% mode: latex
%%% TeX-master: "dokumente"
%%% End: 
